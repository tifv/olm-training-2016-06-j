% $date: 2016-06-11
% $timetable:
%   g9r2:
%     2016-06-11:
%       2:

\section*{Вписанные углы}

% $authors:
% - Фёдор Константинович Нилов

\begin{problems}

\item
На~стороне $AB$ (на~прямой!) равнобедренного треугольника $ABC$ ($AB = BC$)
выбрана точка~$D$.
Через точку~$D$ провели касательную к~описанной окружности треугольника $ADC$.
Она пересекла описанную окружность треугольника $BDC$ в~точке~$M$.
Докажите, что $BM \parallel AC$.

\item
На~прямых~$AB$, $BC$ и~$CA$ лежат точки $C_1$, $A_1$ и~$B_1$ соответственно.
Точка~$O$ такова, что точки $O$, $A$, $B_1$, $C_1$ лежат на~одной окружности
и~точки $O$, $B$, $A_1$ и~$C_1$ лежат на~одной окружности.
Докажите, что точки $O$, $C$, $A_1$ и~$B_1$ также лежат на~одной окружности.

\item
Окружности $S_1$ и~$S_2$ пересекаются в~точке~$A$.
Через точку~$A$ проведена прямая, пересекающая $S_1$ в~точке~$B$, $S_2$
в~точке~$C$.
В~точках $C$ и~$B$ проведены касательные к~окружностям, пересекающиеся
в~точке~$D$.
Докажите, что угол $BDC$ не~зависит от~выбора прямой, проходящей через $A$.

\item
Окружности $S_1$ и~$S_2$ пересекаются в~точках $A$ и~$P$.
Через точку~$A$ проведена касательная~$AB$ к~окружности~$S_1$, а~через
точку~$P$~--- прямая~$CD$, параллельная $AB$
(точки $B$ и~$C$ лежат на~$S_2$, точка~$D$~--- на~$S_1$).
Докажите, что $ABCD$~--- параллелограмм.

\item
Две окружности с~центрами $O_1$ и~$O_2$ пересекаются в~точках $A$ и~$B$.
Описанная окружность треугольника $O_1 B O_2$ пересекает вторую окружность
в~точке~$P$, отличной от~$B$.
Докажите, что точки $O_1$, $A$ и~$P$ лежат на~одной прямой.

\item
В~параллелограмме $ABCD$ на~диагонали~$AC$ лежит точка~$M$ такая, что
четырехугольник $BCDM$~--- вписанный.
Докажите, что прямая~$BD$~--- общая касательная к~описанным окружностям
треугольников $ABM$ и~$ADM$.

\item
Окружности $\omega_1$ и~$\omega_2$ пересекаются в~точках $A_1$ и~$B_1$,
окружности $\omega_2$ и~$\omega_3$ пересекаются в~точках $A_2$ и~$B_2$,
окружности $\omega_3$ и~$\omega_4$ пересекаются в~точках $A_3$ и~$B_3$,
окружности $\omega_4$ и~$\omega_1$ пересекаются в~точках $A_4$ и~$B_4$.
Докажите, что если все $A_i$ лежат на~одной окружности, то~все $B_i$ тоже
лежат на~одной окружности.

\item
В~треугольнике $ABC$ проведена высота~$AH$.
В~треугольниках $ACH$ и~$ABH$ проведены высоты $HM$ и~$HN$ соответственно.
Докажите, что треугольники $MAN$ и~$BAC$ подобны.

\item
Треугольник $ABC$ вписан в~окружность с~центром~$O$.
Прямые $AC$ и~$BC$ вторично пересекают окружность, проходящую через точки~$A$,
$O$ и~$B$, в~точках $E$ и~$K$.
Докажите, что прямые $OC$ и~$EK$ перпендикулярны.

\item
Дан треугольник $ABC$.
Из~точки $D$ на~описанной окружности провели перпендикуляры на~стороны
треугольника $D A_1$, $D B_1$ и~$D C_1$.
Докажите, что точки $A_1$, $B_1$, $C_1$ лежат на~одной прямой.

\item
\subproblem
На~плоскости проведены 4~прямые общего положения.
Докажите, что описанные окружности четырех треугольников, образованных этими
прямыми пересекаются в~одной точке.
\\
\subproblem
На~плоскости проведены 5~прямых общего положения.
Для каждой четверки прямых отмечена точка пересечения четырех описанных
окружностей треугольников.
Докажите, что 5 получившихся точек лежат на~одной окружности.

\end{problems}

