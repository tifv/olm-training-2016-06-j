% $date: 2016-06-13
% $timetable:
%   g9r2:
%     2016-06-13:
%       1:

\subsection*{Добавка}
% $caption: Вписанные углы. Добавка

% $authors:
% - Фёдор Константинович Нилов

% $matter[-contained,no-header]:
% - verbatim: \worksheet*{Вписанные углы}
% - .[contained,fixproblem]
% $matter[contained,no-header,-fixproblem]:
% - verbatim: \resetproblem
% - .[fixproblem]

\begin{problems}

\item
Точка~$H$~--- ортоцентр треугольника $ABC$.
Точку~$P$ на~описанной окружности треугольника $ABC$ отразили относительно
сторон $AB$, $BC$, $CA$ и~получили точки $P_C$, $P_A$, $P_B$ соответственно.
Докажите, что $P_A$, $P_B$, $P_C$ и~$H$ лежат на~одной прямой.

\item
В~окружности~$\alpha$ проведена хорда~$AB$.
Окружность~$\beta$ касается хорды~$AB$ в~точке~$T$ и~пересекает
окружность~$\alpha$ в~точках $X$, $Y$.
Пусть $AX$ и~$TY$ пересекаются в~$M$, а~$BY$ и~$TX$ в~точке~$N$.
Докажите, что $MN \parallel AB$.

\item
В~треугольнике $ABC$ угол при вершине~$B$ равен $60^{\circ}$.
$A A_1$ и~$C C_1$ — биссектрисы треугольника $ABC$.
Докажите, что точка, симметричная $B$ относительно $A_1 C_1$, лежит
на~прямой $AC$.

\end{problems}

