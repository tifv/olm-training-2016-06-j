% $date: 2016-06-04
% $timetable:
%   g8r1:
%     2016-06-04:
%       2:

\section*{Отрезки}

% $authors:
% - Леонид Андреевич Попов

\begin{problems}

\item
На~катетах $AC$ и~$BC$ равнобедренного прямоугольного треугольника отметили
точки $M$ и~$L$ соответственно так, что $MC = BL$.
Точка~$K$~--- середина гипотенузы~$AB$.
Докажите, что треугольник $MKL$ также является прямоугольным равнобедренным.

\item
В~четырехугольнике $ABCD$ верно, что $AD = AB + CD$.
Кроме того, оказалось, что биссектриса угла $A$ проходит через точку~$M$,
середину стороны~$BC$.
Докажите, что биссектриса угла~$D$ также проходит через точку~$M$.

\item
В~треугольнике $ABC$ биссектриса~$AE$ равна по~длине отрезку~$EC$.
Причем $2 AB = AC$.
Найдите углы треугольника $ABC$.

\item
В~трапеции $ABCD$ с~основаниями $AD$ и~$BC$ выполнено $\angle ABD = 90^{\circ}$
и~$AD = BC + CD$.
Найдите отношение $BC$ к~$AD$.

\item
В~равнобедренном треугольнике $ABC$ выполнено $\angle A = 100^{\circ}$.
Докажите, что если $BD$~--- биссектриса угла $B$, то~$BD + DA = BC$.

\item
На~катетах $AC$ и~$BC$ прямоугольного треугольника $ABC$ отметили
точки $K$ и~$L$ соответственно так, что $AK = BC$ и~$CK = BL$.
Найдите угол между прямыми $BK$ и~$AL$.

\end{problems}

