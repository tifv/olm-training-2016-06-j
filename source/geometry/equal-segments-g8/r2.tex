% $date: 2016-06-05
% $timetable:
%   g8r2:
%     2016-06-05:
%       2:

\section*{Отрезки}

% $authors:
% - Леонид Андреевич Попов

\begin{problems}

\item
На~катетах $AC$ и~$BC$ равнобедренного прямоугольного треугольника отметили
точки $M$ и~$L$ соответственно так, что $MC = BL$.
Точка~$K$~--- середина гипотенузы~$AB$.
Докажите, что треугольник $MKL$ также является прямоугольным равнобедренным.

\item
На~стороне~$AC$ треугольника $ABC$ нашлись такие точки $K$ и~$L$, что
$L$~--- середина $AK$ и~$BK$~--- биссектриса угла $LBC$.
Кроме того, оказалось, что $2 BL = BC$.
Докажите, что $KC = AB$.

\item
В~четырехугольнике $ABCD$ верно, что $AD = AB + CD$.
Кроме того, оказалось, что биссектриса угла $A$ проходит через точку~$M$,
середину стороны~$BC$.
Докажите, что биссектриса угла~$D$ также проходит через точку~$M$.

\item
Через точку~$Y$ на~стороне~$AB$ равностороннего треугольника $ABC$ провели
прямую, пересекающую луч~$CA$ за~точкой~$A$ в~точке~$X$ и~сторону~$BC$
в~точке~$Z$.
Оказалось, что $XY = YZ$
и~$AY = BZ$.
Докажите, что $XZ \perp BC$.

\item
В~треугольнике $ABC$ биссектриса~$AE$ равна по~длине отрезку~$EC$.
Причем $2 AB = AC$.
Найдите углы треугольника $ABC$.

\item
В~равнобедренном треугольнике $ABC$ выполнено $\angle A = 100^{\circ}$.
Докажите, что если $BD$~--- биссектриса угла $B$, то~$BD + DA = BC$.

\item
В~параллелограмме $ABCD$ опустили перпендикуляр~$BH$ на~сторону~$AD$.
На~отрезке~$BH$ отметили точку~$M$, равноудалённую от~точек $C$ и~$D$.
Пусть точка~$K$~--- середина стороны~$AB$.
Докажите, что угол $MKD$ прямой.

\item
В~равнобедренном треугольнике $ABC$ ($AB = BC$) на~$BC$ отложили отрезок~$CM$,
равный высоте~$AH$.
На~стороне~$AB$ отметили точку~$K$ так, что $\angle KMC = 90^{\circ}$.
Найдите $\angle ACK$.

\item
Точки $M$ и~$N$~--- середины боковых сторон $AB$ и~$CD$ трапеции $ABCD$.
Перпендикуляр, опущенный из~точки~$M$ на~диагональ~$AC$, и~перпендикуляр,
опущенный из~точки~$N$ на~диагональ~$BD$, пересекаются в~точке~$P$.
Докажите, что $PA = PD$.

\end{problems}

