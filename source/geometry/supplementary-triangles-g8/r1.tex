% $date: 2016-06-03
% $timetable:
%   g8r1:
%     2016-06-03:
%       2:

\section*{Четвертый <<признак равенства>> треугольников}

% $authors:
% - Леонид Андреевич Попов


\begin{problems}

\item
В~выпуклом четырехугольнике $ABCD$:
$AD = BC$; $\angle ABD + \angle CDB = 180^{\circ}$.
Докажите, что $\angle BAD = \angle BCD$.

\item
Пусть $K$~--- середина стороны~$BC$ треугольника $ABC$.
На~лучах $AB$ и~$AC$ взяты точки $X$ и~$Y$ соответственно таким образом, что
$AX = AY$ и~точка $K$ лежит на~отрезке~$XY$.
Докажите, что $BX = CY$.

\item
На~стороне~$BC$ равностороннего треугольника $ABC$ взята точка~$M$,
а~на~продолжении стороны~$AC$ за~точку~$C$~--- точка~$N$, причем $AM = MN$.
Докажите, что $BM = CN$.

\item
В~выпуклом четырехугольнике $ABCD$, в~котором $AB = CD$, на~сторонах $AB$
и~$CD$ выбраны точки $K$ и~$M$ соответственно.
Оказалось, что $AM = KC$, $BM = KD$.
Докажите, что угол между прямыми $AB$ и~$KM$ равен углу между
прямыми $KM$ и~$CD$.

\item
В~неравнобедренном треугольнике $ABC$ биссектрисы $A A_1$ и~$B B_1$ пресекаются
в~точке~$I$.
Найдите $\angle C$, если $A_1 I = B_1 I$.

\item
Обязательно~ли треугольник равнобедренный, если точка пересечения биссектрис
одинаково удалена от~середин двух сторон?

\item
В~параллелограмме $ABCD$ опустили перпендикуляр~$BH$ на~сторону~$AD$.
На~отрезке~$BH$ отметили точку~$M$, равноудаленную от~точек $C$ и~$D$.
Пусть точка~$K$~--- середина стороны~$AB$.
Докажите, что угол $MKD$ прямой.

\item
В~равнобедренном треугольнике $ABC$ ($AB = BC$) на~$BC$ отложили отрезок~$CM$
равный высоте~$AH$.
На~стороне~$AB$ отметили точку~$K$ так, что $\angle KMC = 90^{\circ}$.
Найти $\angle ACK$.

\item
Точки $M$ и~$N$~--- середины боковых сторон $AB$ и~$CD$ трапеции $ABCD$.
Перпендикуляр, опущенный из~точки~$M$ на~диагональ~$AC$, и~перпендикуляр,
опущенный из~точки~$N$ на~диагональ~$BD$, пересекаются в~точке~$P$.
Докажите, что $PA = PD$.

\end{problems}

