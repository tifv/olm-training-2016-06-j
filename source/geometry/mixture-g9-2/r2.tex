% $date: 2016-06-08
% $timetable:
%   g9r2:
%     2016-06-08:
%       1:

\worksheet*{Геометрический винегрет}

% $authors:
% - Фёдор Константинович Нилов

\begin{problems}

\item
В~треугольнике $ABC$ биссектриса $AK$ перпендикулярна медиане $CL$.
Докажите, что в~треугольнике $BKL$ также одна из~биссектрис перпендикулярна
одной из~медиан.

\item
Квадрат и~прямоугольник одинакового периметра имеют общий угол.
Докажите, что точка пересечения диагоналей прямоугольника лежит на~диагонали
квадрата.

\item
Дан треугольник $ABC$.
На~продолжениях сторон~$AB$ и~$CB$ за~точку~$B$ взяты точки $C_1$ и~$A_1$
соответственно так, что $AC = A_1 C = A C_1$.
Докажите, что описанные окружности треугольников $A B A_1$ и~$C B C_1$
пересекаются на~биссектрисе угла~$B$.

\item
Дан правильный треугольник $ABC$.
Рассматриваются всевозможные прямые~$l$, проходящие через вершину~$B$ и~лежащие
вне треугольника.
Окружность~$\alpha$ касается стороны~$AB$, продолжения стороны~$AC$
за~точку~$A$ и~прямой~$l$.
Окружность~$\beta$ касается стороны~$BC$, продолжения стороны~$AC$ за~точку~$C$
и~прямой~$l$.
Докажите, что сумма радиусов окружностей $\alpha$ и~$\beta$ не~зависит
от~прямой $l$.

\item
Из~листа бумаги в~клетку вырезали квадрат $2 \times 2$.
Используя только линейку без делений и~не~выходя за~пределы квадрата, разделите
его диагональ на~$6$ равных частей.

\item
На~сторонах треугольника $ABC$ во~внешнюю сторону построили правильные
треугольники $A B C_1$, $B C A_1$ и~$C A B_1$.
На~отрезке $A_1 B_1$ во~внешнюю сторону треугольника $A_1 B_1 C_1$ построен
правильный треугольник $A_1 B_1 C_2$.
Докажите, что $C$~--- середина отрезка $C_1 C_2$.

\item
В~остроугольном треугольнике один из~углов равен $60^{\circ}$.
Докажите, что прямая, проходящая через центр описанной окружности и~точку
пересечения медиан, отсекает от~него равносторонний треугольник.

\item
Дан равнобедренный треугольник $ABC$ с~углом~$\alpha$ при вершине~$A$.
Дуга~$BC$ с~градусной мерой~$\beta$ построена вовсе треугольника.
Нашлись два луча, проходящие через вершину~$B$, которые делят сторону~$BC$
и~дугу~$BC$ на~три равные части.
Найдите отношение $\alpha$ и~$\beta$.

\end{problems}

