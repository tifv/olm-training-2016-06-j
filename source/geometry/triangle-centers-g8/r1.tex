% $date: 2016-06-06
% $timetable:
%   g8r1:
%     2016-06-06:
%       1:

\section*{Центры треугольников}

% $authors:
% - Леонид Андреевич Попов

\begin{problems}

\item
В~треугольнике $ABC$ угол~$A$ равен $\alpha$.
Найдите
\\
\subproblem угол между высотами $B H_B$ и~$C H_C$;
\\
\subproblem угол между биссектрисами $B B_1$ и~$C C_1$.

\item
В~четырехугольнике $ABCD$ углы $B$ и $C$ равны по $146^{\circ}$.
Биссектриса угла~$D$ пересекает серединный перпендикуляр к~стороне~$BC$
в~точке~$O$.
Найдите $\angle AOD$.

\item
В~прямоугольнике $ABCD$ биссектрисы угла~$B$ и~внешнего угла~$D$ пересекают
сторону~$AD$ и~прямую~$AB$ в~точках $K$, $M$ соответственно.
Докажите что отрезок~$KM$ равен и~перпендикулярен отрезку~$BD$.

%\item
%В~выпуклом шестиугольнике $ABCDEF$, все углы которого тупые,
%$\angle A = \angle B$, $\angle C = \angle D$, $\angle E = \angle F$.
%Докажите, что серединные перпендикуляры к~его сторонам $AB$, $CD$, $EF$
%пересекаются в~одной точке.

\item
В~треугольнике $ABC$ сторона~$AC$ наименьшая.
На~сторонах $AB$ и~$CB$ взяты точки $K$ и~$L$ соответственно, причем
$KA = AC = CL$.
Пусть $M$~--- точка пересечения $AL$ и~$KC$, а~$I$~--- точка пересечения
биссектрис треугольника $ABC$.
Докажите, что прямая~$MI$ перпендикулярна прямой~$AC$.

\item
Биссектрисы двух соседних углов четырехугольника пересекаются в~середине его
стороны.
Докажите, что либо у~этого четырехугольника равны два угла, либо две стороны
параллельны.

\item
В~прямоугольнике $ABCD$ точка~$M$~--- середина стороны~$CD$.
Через точку~$C$ провели прямую, перпендикулярную прямой~$BM$, а~через
точку~$M$~--- прямую, перпендикулярную диагонали~$BD$.
Докажите, что два проведенных перпендикуляра пересекаются на~прямой~$AD$.

\item
На~сторонах $AB$ и~$AD$ единичного квадрата $ABCD$ выбраны точки $N$ и~$P$
соответственно.
Причем периметр треугольника $ANP$ равен $2$.
Докажите, что $\angle NCP = 45^{\circ}$.

\item
На~сторонах $AB$ и~$AD$ квадрата $ABCD$ выбраны точки $N$ и~$P$ соответственно,
а~на~отрезке~$AN$ выбрана точка~$Q$ так, что $NP = NC$
и~$\angle QPN = \angle NCB$.
Докажите, что $\angle BCQ = \frac{1}{2} \angle AQP$.

\end{problems}

