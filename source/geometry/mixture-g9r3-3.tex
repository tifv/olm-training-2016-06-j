% $date: 2016-06-12
% $timetable:
%   g9r3:
%     2016-06-12:
%       2:

\worksheet*{Геометрическая добавка}

% $authors:
% - Фёдор Константинович Нилов

\begin{problems}

\item
Дан треугольник $ABC$.
Из~точки~$D$ на~описанной окружности провели перпендикуляры на~стороны
треугольника $D A_1$, $D B_1$ и~$D C_1$.
Докажите, что точки $A_1$, $B_1$, $C_1$ лежат на~одной прямой.

\item
На~плоскости проведены 4~прямые общего положения.
Докажите, что описанные окружности четырех треугольников, образованных этими
прямыми пересекаются в~одной точке.

\item
\subproblem
Дан треугольник $ABC$.
Пусть $H$~--- ортоцентр, $A_1$, $B_1$, $C_1$~--- середины сторон треугольника,
$A_2$, $B_2$, $C_2$~--- середины отрезков $AH$, $BH$, $CH$,
а~точки $A_3$, $B_3$, $C_3$~--- основания высот треугольника $ABC$.
Докажите, что точки
$A_1$, $B_1$, $C_1$, $A_2$, $B_2$, $C_2$, $A_3$, $B_3$, $C_3$ лежат
на~одной окружности.
\\
\subproblem
Докажите, что среди трех дуг $A_1 A_3$, $B_1 B_3$, $C_1 C_3$ одна равна сумме
двух других.

\item
Дан треугольник $ABC$.
Пусть $A_1$, $B_1$~--- середины высот, проведенных из~точек $A$, $B$,
точка~$C'$~--- основание высоты из~точки~$C$, $H$~--- ортоцентр.
Докажите, что точки $A_1$, $B_1$, $C'$, $H$ лежат на~одной окружности.

\item
Докажите, что площадь правильного восьмиугольника равна произведению длин
наибольшей и~наименьшей его диагоналей.

\item
Дан тортик в~форме буквы~\textsf{L}\,.
С~помощью одной линейки проведите прямую, которая
делит его площадь пополам.

\item
На~стороне треугольника во~внешнюю сторону построена дуга.
Рассмотрим фигуру, являющуюся объединением данного треугольника
и~образовавшегося сектора.
Постройте с~помощью циркуля и~линейки прямую через середину дуги, которая
делит площадь данной фигуры пополам.

\end{problems}

