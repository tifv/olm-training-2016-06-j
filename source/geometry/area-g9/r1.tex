% $date: 2016-06-12
% $timetable:
%   g9r1:
%     2016-06-12:
%       1:

\worksheet*{Площади}

% $authors:
% - Фёдор Константинович Нилов

\begin{problems}

\item
В~трапеции $ABCD$ с~меньшим основанием~$BC$ через точку~$B$ провели прямую,
параллельную стороне~$CD$.
Эта прямая пересекла диагональ~$AC$ в~точке~$E$.
Докажите, что площади треугольников $ABC$ и~$CED$ равны.

\item
Дан тортик в~форме буквы~$L$.
С~помощью одной линейки проведите прямую, которая
делит его площадь пополам.

\item
В~правильном шестиугольнике $ABCDEF$ точки $K$ и~$L$~---
середины сторон $AB$ и~$BC$ соответственно.
Отрезки $DK$ и~$EL$ пересекаются в~точке~$N$.
Докажите, что площадь четырехугольника $KBLN$ равна площади треугольника $DEN$.

\item
Докажите, что площадь правильного восьмиугольника равна произведению длин
наибольшей и~наименьшей его диагоналей.

\item
Шестиугольник $ABCDEF$ вписан в~окружность.
Диагонали $AD$, $BE$ и~$CF$ являются диаметрами этой окружности.
Докажите, что площадь шестиугольника $ABCDEF$ равна удвоенной площади
треугольника $ACE$.

\item
Внутри выпуклого равноугольного многоугольника взяли произвольную точку
и~посчитали сумму расстояний до~сторон.
Докажите, что данная сумма не~зависит от~выбора точки внутри многоугольника.

\item
На~стороне треугольника во~внешнюю сторону построена дуга.
Рассмотрим фигуру, являющуюся объединением данного треугольника
и~образовавшегося сектора.
Постройте с~помощью циркуля и~линейки прямую через середину дуги, которая делит
площадь данной фигуры пополам.

\item
В~остроугольном треугольнике провели высоты $AD$ и~$CE$.
Построили квадрат $ACPQ$ и~прямоугольники $CDMN$ и~$AEKL$, у~которых $AL = AB$
и~$CN = CB$.
Докажите, что площадь квадрата $ACPQ$ равна сумме площадей
прямоугольников $CDMN$ и~$AEKL$.

\item
Дан выпуклый шестиугольник $ABCDEF$, у~которого $AB$ параллельно $DE$, $BC$
параллельно $EF$, $CD$ параллельно $AF$.
Докажите, что площади треугольников $ACE$ и~$BDF$ равны.

\item
Стороны выпуклого четырехугольника разделили на~8 равных частей.
Соответствующие точки деления на~противоположных сторонах соединены друг
с~другом, полученные клетки раскрашены в~шахматном порядке.
Докажите, что сумма площадей черных клеток равна сумме площадей белых клеток.

\item
Дан выпуклый пятиугольник.
Каждая диагональ отсекает от~него треугольник.
Докажите, что сумма площадей пяти данных треугольников больше площади данного
пятиугольника.

\end{problems}

