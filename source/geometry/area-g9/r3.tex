% $date: 2016-06-10
% $timetable:
%   g9r3:
%     2016-06-10:
%       1:

\section*{Площади}

% $authors:
% - Фёдор Константинович Нилов

\begin{problems}

\item
Дан треугольник $ABC$.
Внутри него взята такая точка~$M$, что площади треугольников
$ABM$, $ACM$, $BCM$ равны.
Докажите, что $M$~--- точка пересечения медиан треугольника $ABC$.

\item
Дан треугольник $ABC$.
На~сторонах $BC$, $AC$, $AB$ отметили точки $A'$, $B'$, $C'$ так, что
$AB' : B'C = CA' : A'B = BC' : C'A = 2 : 3$.
Найдите отношение площадей треугольников $A'B'C'$ и~$ABC$.

\item
Дан параллелограмм $ABCD$.
Внутри него дана точка~$M$.
Докажите, что сумма площадей треугольников $ABM$ и~$CDM$ равна сумме площадей
треугольников $BCM$ и~$ADM$.

\item
Дана трапеция $ABCD$, $AB = 10$.
Обозначим точку пересечения $AC$ и~$BD$ через $X$.
На~основании~$AD$ отмечена точка~$Y$ такая, что $XY \parallel CD$.
Расстояние от~$X$ до~$AB$ равно 4, а~расстояние от~$Y$ до~$CD$ равно $5$.
Найдите $CD$.

\item
\subproblem
Дан правильный треугольник $ABC$.
Внутри него дана произвольная точка~$M$.
Обозначим расстояния от~точки~$M$ до~сторон треугольника $ABC$ через
$d_{a}$, $d_{b}$, и~$d_{c}$.
Докажите, что сумма $d_{a} + d_{b} + d_{c}$ не~зависит от~выбора точки~$M$.
\\
\subproblem
Докажите, что сумма расстояний от~произвольной точки внутри выпуклого
\emph{равностороннего} многоугольника до прямых, содержащих его стороны,
не~зависит от~выбора точки.
\\
\subproblem
То же для \emph{равноугольного} многоугольника.

\item
Дан четырехугольник $ABCD$, в~котором диагонали $AC$ и~$BD$ равны
и~пересекаются в~точке~$O$.
Внутри него дана точка~$M$ такая, что $BM$ параллельно $CD$, а~$CM$
параллельно $AD$.
Докажите, что $M$ лежит на~биссектрисе угла $AOD$.

\item
Дан четырехугольник $ABCD$.
Обозначим середины его диагоналей $AC$ и~$BD$ через $M$ и~$N$ соответственно.
Пусть точка пересечения диагоналей $ABCD$ лежит на~отрезках $BN$ и~$CM$.
Найдите отношение площадей четырехугольников $BMNC$ и~$ABCD$.

\item
Дан правильный пятиугольник $ABCDE$.
Пусть $M$~--- середина $BC$, а~$N$~--- середина $CD$.
Отрезки $AM$ и~$BN$ пересекаются в~точке~$O$.
Докажите, что площадь треугольника $ABO$ равна площади треугольника $CMON$.

\item
Дан треугольник $ABC$.
В~нем провели медианы $m_{a}$, $m_{b}$, $m_{c}$.
Из~отрезков $m_{a}$, $m_{b}$, $m_{c}$ составили треугольник.
Найдите отношение площадей полученного треугольника и~треугольника $ABC$.

\end{problems}

