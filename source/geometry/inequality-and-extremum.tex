% $date: 2016-06-04
% $timetable:
%   g9r3:
%     2016-06-05:
%       1:
%   g9r2:
%     2016-06-04:
%       1:
%   g9r1:
%     2016-06-05:
%       2:

\worksheet*{Геометрические неравенства и экстремумы}

% $authors:
% - Александр Давидович Блинков

\begin{problems}

\item
\subproblem
Каждая из~высот параллелограмма не~меньше той стороны, которой она
перпендикулярна.
Найдите угол между диагоналями параллелограмма.
\\
\subproblem
Определите вид треугольника, у~которого каждая из~двух высот не~меньше стороны,
к~которой она проведена.

\item
Сумма стороны и~проведенной к~ней высоты
\\
\subproblem в~параллелограмме
\quad
\subproblem в~треугольнике
\\
одна и~та~же для всех сторон.
Определите вид параллелограмма/треугольника.

\item
На~сторонах $BC$ и~$AC$ равностороннего треугольника $ABC$ отмечены
точки $X$ и~$Y$ соответственно.
Докажите, что из~отрезков $AX$, $BY$ и~$XY$ можно составить треугольник.

\item
\subproblem
Остроугольный треугольник расположен внутри некоторой окружности.
Докажите, что ее радиус не~меньше радиуса окружности, описанной около
треугольника.
\\
\subproblem
Верно~ли аналогичное утверждение для прямоугольного или тупоугольного
треугольников?

\item
На~сторонах $AB$, $BC$ и~$AC$ треугольника $ABC$ отмечены
точки $C'$, $A'$ и~$B'$ соответственно так, что угол $A'C'B'$~--- прямой.
Докажите, что отрезок $A'B'$ длиннее диаметра вписанной окружности
треугольника $ABC$.

\item
Две окружности пересекаются в~точках $P$ и~$Q$.
Точка~$A$ лежит на~первой окружности, но~не~вне второй.
Прямые $AP$ и~$AQ$ пересекают вторую окружность в~точках $B$ и~$C$.
При каком положении точки~$A$ треугольник $ABC$ имеет наибольшую площадь?

\item
Докажите, что из~всех треугольников с~данным углом и~данным периметром
наибольшую площадь имеет равнобедренный.

\item
На~стороне~$BE$ равностороннего треугольника $ABE$ вне его построен
ромб $BCDE$.
Отрезки $AC$ и~$BD$ пересекаются в~точке~$F$.
Докажите, что $AF < BD$.

\item
Известно, что в~неравностороннем треугольнике $ABC$ точка, симметричная точке
пересечения медиан относительно стороны~$BC$, принадлежит описанной окружности.
Докажите, что $\angle BAC < 60^{\circ}$.

\end{problems}

