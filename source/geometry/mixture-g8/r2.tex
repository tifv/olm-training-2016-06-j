% $date: 2016-06-10
% $timetable:
%   g8r2:
%     2016-06-10:
%       1:

\section*{Разнобой}

% $authors:
% - Анна Николаевна Доледенок

\begin{problems}

\item
Два одинаковых прямоугольных треугольника из~бумаги удалось положить один
на~другой так, как показано на~рисунке (при этом вершина прямого угла одного
попала на~сторону другого).
Докажите, что заштрихованный треугольник равносторонний.
\begin{center}
\jeolmfigure[width=0.15\textwidth]{triangles}
\end{center}

\item
Высоты $AA'$ и~$BB'$ треугольника $ABC$ пересекаются в~точке~$H$.
Точки $X$ и~$Y$~--- середины отрезков $AB$ и~$CH$ соответственно.
Докажите, что прямые $XY$ и~$A'B'$ перпендикулярны.

\item
В~равнобедренном треугольнике $ABC$ с~основанием~$BC$ проведена
биссектриса~$B B_1$.
Оказалось, что $B C = A B_1$.
Докажите, что $B C = A B_1 = B B_1$.

\item
Точка~$K$~--- середина гипотенузы~$AB$ прямоугольного треугольника $ABC$.
На~катетах $AC$ и~$BC$ выбраны точки $M$ и~$N$ соответственно так, что угол
$\angle MKN$~--- прямой.
Докажите, что из~отрезков $AM$, $BN$ и~$MN$ можно составить прямоугольный
треугольник.

\item
Точка~$M$~--- середина стороны~$AC$ треугольника $ABC$.
Точка~$D$ на~стороне~$BC$ такова, что $\angle BMA = \angle DMC$.
Оказалось, что $CD + DM = BM$.
Докажите, что $\angle ACB + \angle ABM = \angle BAC$.

\item
В~треугольнике $ABC$ проведены биссектрисы $B B_1$ и~$C C_1$.
Найдите угол~$A$, если известно, что $\angle C_1 B_1 B = 30^{\circ}$,
а~$\angle C \neq 120^{\circ}$.

\end{problems}

