% $date: 2016-06-03
% $timetable:
%   g9r3:
%     2016-06-04:
%       2:
%   g9r2:
%     2016-06-03:
%       2:
%   g9r1:
%     2016-06-03:
%       1:

\section*{Вокруг вписанной и вневписанной окружностей}

% $authors:
% - Александр Давидович Блинков

\begin{problems}

\item
Докажите, что в~любом треугольнике:\\
\subproblem
$1 / r = 1 / r_{a} + 1 / r_{b} + 1 / r_{c}$;
\qquad
\subproblem
$S = \sqrt{r \cdot r_{a} \cdot r_{b} \cdot r_{c}}$.

\item
Докажите, что треугольник является прямоугольным тогда и~только тогда, когда:
\\
\subproblem
площадь треугольника равна произведению радиусов вписанной и~одной
из~вневписанных окружностей;
\\
\subproblem
площадь треугольника равна произведению двух радиусов вневписанных окружностей.

\item
На~сторонах $AB$, $BC$, $CD$ и~$DA$ единичного квадрата $ABCD$ отметили точки
$K$, $L$, $M$ и~$N$ соответственно так, что прямые $KM$ и~$LN$ параллельны
сторонам квадрата.
Отрезок~$KL$ отсекает от~квадрата треугольник периметра~$1$.
Треугольник какой площади отсекает от~квадрата отрезок~$MN$?

\item
Даны квадрат $ABCD$ и окружность с~центром в~вершине~$A$, проходящая через
вершины $B$~и~$D$.
Точка~$K$ делит сторону~$BC$ в~отношении $1 : 2$, считая от~вершины~$B$.
В~каком отношении делит сторону~$CD$ касательная к~окружности, проведенная
из~точки~$K$?

\item
Дан треугольник $ABC$.
На~лучах $AB$ и~$AC$ (вне треугольника) построены точки $A_1$ и~$A_2$
соответственно так, что $B A_1 = C A_2 = BC$.
$A_0$~--- точка пересечения отрезков $B A_2$ и~$C A_1$.
Докажите, что прямая, проходящая через $A_0$ перпендикулярно прямой~$BC$,
содержит центр вневписанной окружности треугольника $ABC$.

\item
$ABCD$~--- параллелограмм.
Вневписанные окружности треугольников $ABC$ и~$ACD$ касаются сторон $BC$ и~$CD$
соответственно.
Докажите, что точки их касания с~прямой~$AC$ совпадают.

\item
Дан параллелограмм $ABCD$.
Вневписанная окружность треугольника $ABD$ касается продолжений сторон $AD$
и~$AB$ в~точках $M$ и~$N$.
Докажите, что точки пересечения отрезка~$MN$ со~сторонами $BC$ и~$CD$ лежат
на~вписанной окружности треугольника $BCD$.

\item
Окружность, вписанная в~прямоугольный треугольник, касается катета~$BC$
и~гипотенузы~$AB$ в~точках $P$ и $Q$ соответственно.
Вневписанная окружность, касающаяся катета~$BC$, касается продолжения
катета~$AC$ в~точке~$T$.
Докажите, что точки $P$, $Q$ и~$T$ лежат на~одной прямой.

\item
Вневписанная окружность треугольника $ABC$ касается его стороны~$BC$
в~точке~$K$, а~продолжения стороны~$AB$~--- в~точке~$L$.
Другая вневписанная окружность касается продолжений сторон $AB$ и~$AC$
в~точках $M$ и~$N$ соответственно.
Прямые $KL$ и~$MN$ пересекаются в~точке~$X$.
Докажите, что $CX$~--- биссектриса угла $ACN$.

\end{problems}

