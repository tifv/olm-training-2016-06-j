% $date: 2016-06-03
% $timetable:
%   g8r1:
%     2016-06-03:
%       2:
%   g8r2:
%     2016-06-05:
%       2:

\worksheet*{Задачи для самостоятельных размышлений}
% $caption: Разнобой (геометрия)

% $authors:
% - Леонид Андреевич Попов

\begin{problems}

\item
В~параллелограмме $ABCD$ опустили перпендикуляр~$BH$ на~сторону~$AD$.
На~отрезке~$BH$ отметили точку~$M$, равноудаленную от~точек $C$ и~$D$.
Пусть точка~$K$~--- середина стороны~$AB$.
Докажите, что угол $MKD$ прямой.

\item
В~равнобедренном треугольнике $ABC$ ($AB = BC$) на~$BC$ отложили отрезок~$CM$
равный высоте~$AH$.
На~стороне~$AB$ отметили точку~$K$ так, что $\angle KMC = 90^{\circ}$.
Найти $\angle ACK$.

\item
Точки $M$ и~$N$~--- середины боковых сторон $AB$ и~$CD$ трапеции $ABCD$.
Перпендикуляр, опущенный из~точки~$M$ на~диагональ~$AC$, и~перпендикуляр,
опущенный из~точки~$N$ на~диагональ~$BD$, пересекаются в~точке~$P$.
Докажите, что $PA = PD$.

\end{problems}

