% $date: 1--4 июня 2016

\section*{Суммируем все, что движется}

% $authors:
% - Глеб Александрович Погудин

% $style[-announcement]:
% - .[announcement]

Многие из~вас знают формулу сумму арифметической прогресии, а~некоторые даже
и~формулу суммы геометрической.
Кроме того, наверняка в~листочках по~математической индукции вы видели
разлапистые формулы для суммы квадратов первых $n$~чисел и~для суммы кубов
первых $n$~чисел:
\begin{align*}
    1^2 + 2^2 + 3^2 + \ldots + n^2
&=
    \frac{2n^3 + 3n^2 + n}{6}
\\
    1^3 + 2^3 + 3^3 + \ldots + n^3
&=
    \frac{n^2 (n + 1)^2}{4}
\end{align*}

Их довольно легко, пусть и~не~очень приятно, доказать, если ты их уже как-то
угадал.
Но~как такое угадывать-то?

Вообще, в~математике задача вычисления суммы какой-то последовательности
чисел~--- задача очень важная и~интересная.
На~спецкурсе мы обсудим некоторые способы и~подходы, как это можно делать
во~многих случаях, например, в~случае сумм степеней натуральных чисел.

От~слушателей я ожидаю твердности духа и~несгибаемости воли, так как придется
постоянно лицезреть многоточия и~довольно регулярно что-то считать.

