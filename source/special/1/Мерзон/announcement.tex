% $date: 1--4 июня 2016

\section*{Квадратичные характеры малых чисел}

% $authors:
% - Григорий Александрович Мерзон

% $style[-announcement]:
% - .[announcement]

Со~времен Евклида многие знают, как решать линейные уравнения в~целых числах
($a x + b y = c$) или, что практически то~же самое, в~остатках
($a x \equiv c \mod{b}$).
Естественный следующий шаг~--- решение квадратных уравнений.
Здесь возникает намного более богатая теория~--- которой занимались Ферма,
Эйлер, Лежандр, Гаусс и~другие~--- а~мы увидим только ее отблески.

Мы поговорим о~том, какие остатки по~простому модулю являются квадратами,
а~какие нет.
По~пути будут затронуты классические вопросы теории чисел о~представимости
целых чисел в~виде сумм квадратов и~о~простых числах в~арифметических
прогрессиях.

От~слушателей ожидается, что они уверено владеют арифметикой остатков;
хочется надеяться также, что они что-то слышали про малую теорему Ферма.

