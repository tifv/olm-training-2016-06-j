% $date: 1--4 июня 2016

\worksheet*{Начала теории многогранников}

% $authors:
% - Алексей Вадимович Доледенок

% $style[-announcement]:
% - .[announcement]

Наверняка каждый из~вас хоть раз в~жизни склеивал из~бумаги кубик.
Кубик является простейшим примером многогранника, о~многогранниках и~пойдет
речь на~наших занятиях.

Чем многогранники отличаются от~многоугольников?
Почему склеенный кубик сохраняет свою форму?
Правда~ли, что у~любого многогранника есть развертка?
Что можно склеить из~обычного прямоугольного листочка бумаги?
Является~ли баян многогранником?

На~эти и~многие другие вопросы мы научимся отвечать, попутно узнав, что такое
формула Эйлера, теорема Коши о~жесткости, теорема Александрова, гипотеза
кузнечных мехов.
От~слушателей требуется не~бояться слов <<индукция>>, <<дерево>>,
<<степень вершины>>.

