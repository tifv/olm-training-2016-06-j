% $date: 5--8 июня 2016

\section*{Постулат Бертрана и теорема Чебышёва}

% $authors:
% - Фёдор Львович Бахарев

\begingroup
    \ifdefined\mathup
        \def\pifunc{\mathup{\pi}}%
    \else
        \def\pifunc{\uppi}%
    \fi

\begin{problems}

\item
Докажите, что если число $2^n + 1$ простое, то~$n$~--- степень двойки.

\item
Докажите, что если $n < p < 2 n$ и~$p$ просто, то~$p$ входит в~$C_{2n}^n$
ровно в~первой степени.

\item
Докажите, что для $2 n / 3 < p < n$ число $C_{2n}^n$ не~кратно $p$.

\item
Докажите, что при любом натуральном $n$ имеют место наравенства
\\
\subproblem $C_{2n}^{n} < 4^{n}$;
\qquad
\subproblem $C_{2n}^{n} \geq 4^{n} / (2 \sqrt{n})$;
\qquad
\subproblem $C_{2n}^{n} < 4^{n} / \sqrt{2 n + 1}$.

\end{problems}

Количество простых чисел в промежутке от $1$ до $n$ обозначается через
$\pifunc(n)$.

\begin{problems}

\item
Докажите, что существует натуральное $n > 2$ такое, что
$\pifunc(n) / n < 1 / 1000$.

\item
Докажите, что для $\sqrt{2n} < p$ число $C_{2n}^n$ не~кратно $p^2$.

\item
Докажите, что любой примарный сомножитель, то~есть сомножитель вида $p^{k}$, 
числа $C_{2n}^{n}$ не~превосходит $2 n$.

\item
Докажите, что не~существует многочлена, принимающего только простые значения
при всех натуральных значениях аргумента.

\item
Докажите, что число $1 + 1 / 2 + 1 / 3 + \ldots + 1 / n$ не~является целым
ни~при каком $n > 1$
\\
\subproblem с~помощью постулата Бертрана; 
\\
\subproblem без помощи постулата Бертрана.

\item
Докажите, что произведение всех целых чисел от~$(2^{1917} + 1)$
до~$(2^{1991} - 1)$ включительно не~есть квадрат целого числа
\\
\subproblem пользуясь постулатом Бертрана;
\\
\subproblem без помощи постулата Бертрана.

\item
Докажите, что для любого натурального $n > 4$ между $n$ и~$2 n$ существует
число, представимое в~виде произведения двух различных простых.

\item
Докажите, что для любого натурального $n > 15$ между $n$ и~$ 2n$ существует
число, представимое в~виде произведения трех различных простых.

\item
Докажите, что неравенство $C_{2m+1}^{m+1} \leq 4^{m}$ выполнено при любом
натуральном $n$.

\end{problems}

\endgroup % \def\pifunc

