% $date: 5--8 июня 2016

\section*{Вокруг постулата Бертрана и теоремы Чебышёва}

% $authors:
% - Фёдор Львович Бахарев

% $style[-announcement]:
% - .[announcement]

Всем вам хорошо известно, что последовательность простых чисел
$2$, $3$, $5$, $7$, {\ldots} бесконечна.
Размеры промежутков между соседними числами могут быть сколь угодно большими
(кстати, почему?).
Однако сколь велик может быть промежуток между двумя соседними числами?
Согласно самой известной оценке, этот промежуток не~может быть больше числа,
с~которого он начинается.
Это утверждение и~называют постулатом Бертрана, так как оно было высказано им
в~форме предположения и~проверено вручную для $n < 3000000$.
Впервые оно было доказано российским математиком Пафнутием Львовичем Чебышёвым
в~середине XIX в.
Значительно более простое доказательство нашел индийский гений Рамануджан.
Кроме того, элементарное доказательство нашел математик Поль Эрдёш.
Это доказательство было опубликовано в~статье в~1932 году, когда Эрдёшу было
всего 19 лет.

\begin{quote}

\claim{Постулат Бертрана}
Для каждого натурального $n$ существует такое простое число~$p$, что
$n < p \leq 2 n$.

\end{quote}

Эта теорема открывает путь к~вопросу о~поведении простых чисел.
Ответ на~вопрос с~какой скоростью возрастают простые числа
описывается теоремой Чебышёва, к~формулировке и~доказательству которой
приложили руку такие великие умы, как Гаусс, Лежандр, Риман, Адамар.

Для формулировки теоремы нам придется разобраться с~тем, что такое логарифм
и~хотя~бы приблизительно понять, как описывать рост той или иной
последовательности.
Это потребует некоторого терпения, однако на~выходе мы получим красивый
результат.

От~слушателей предполагается хорошее знание того, что такое простое число.
Неплохо было~бы знать что-то про биномиальные коэффициенты.

