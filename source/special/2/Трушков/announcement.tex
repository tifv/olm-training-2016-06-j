% $date: 5--8 июня 2016

\worksheet*{Разрезаем то, что разрезается}

% $authors:
% - Владимир Викторович Трушков

% $style[-announcement]:
% - .[announcement]

Теорема Бойяи-Гервина гласит, что если есть два многоугольника равной площади,
то~можно один разрезать на~кусочки, из~которых можно сложить второй.
Доказательство этой теоремы конструктивно, но~не~оптимально.
Имеется в~виду то, что количество кусочков, необходимых, например, для того,
чтобы из~правильного шестиугольника сделать квадрат, очень большое.
Мы постараемся сделать разрезание оптимальние, для чего изучим такую забавную
вещь, как мозаики.

