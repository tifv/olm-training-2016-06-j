% $date: 10--13 июня 2016

\section*{Вокруг дерева Фарея}

% $authors:
% - Николай Германович Мощевитин

% $style[-announcement]:
% - .[announcement]

Все вы должны хорошо знать, как надо складывать рациональные дроби, то~есть
правило
\[
    \frac{a}{b} + \frac{c}{d}
=
    \frac{a d + b c}{b d}
\; . \]
Оказывается, иногда уместно <<складывать>> дроби по-другому, вот так:
\[
    \frac{a}{b} \oplus \frac{c}{d}
=
    \frac{a + c}{b + d}
\; . \]
Эта процедура, называющаяся \emph{взятием медианты,} в-частности, позволяет
построить все рациональные числа из~0 и~1 в~качестве вершин бинарного
дерева~--- дерева Фарея.
С~помощью дерева Фарея мы построим теорию цепных дробей и~докажем несколько
теорем о~приближениях.
Также будет рассказано об~основах конечной геометрии
(геометрии, когда плоскость состоит из~конечного числа точек),
которую мы будем использовать.

От слушателей ожидается уверенное владение арифметикой остатков.

