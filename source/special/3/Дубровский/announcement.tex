% $date: 10--13 июня 2016

\worksheet*{Доказательства считаются}

% $authors:
% - Владимир Натанович Дубровский

% $style[-announcement]:
% - .[announcement]

Получить хорошо известную формулу $1 + 2 + \ldots + n = n (n + 1) / 2$ можно,
подсчитывая двумя разными способами количество пар различных чисел от~$1$
до~$n + 1$ или количество пар необязательно различных чисел от~$1$ до~$n$.
Аналогичным образом, т.~е. подсчитывая разными способами число элементов
некоторого множества, можно получить и~формулы похитрее, например, формулу для
суммы кубов:
\[
    1^3 + 2^3 + \ldots + n^3
=
    (1 + 2 + \ldots + n)^2
\]
или разнообразные формулы для чисел Фибоначчи
$f_{0} = 1$, $f_{1} = 1$, $f_{2} = 2$, $f_{3} = 3$, $f_{4} = 5$, \ldots,
такие как
\[
    f_{m+n} = f_{m} \cdot f_{n} + f_{m-1} \cdot f_{n-1}
\]
или более сложная:
\[
    f_{n} + f_{n-1} +
    2^{0} f_{n-2} + 2^{1} f_{n-3} + 2^2 f_{n-4} +
    \ldots +
    2^{n-2} f_{0}
=
    2^n
\, . \]
Будут рассмотрены классические тождества для разных комбинаторных чисел
(числа сочетаний и~т.~п.), арифметические теоремы
(например, малая теорема Ферма) и~многое другое.

Предварительных знаний, кроме знакомства с~классической комбинаторикой,
не~требуется.

