% $date: 10--13 июня 2016

\section*{Лингвистика: языковое разнообразие}

% $authors:
% - Евгения Сергеевна Клягина

% $style[-announcement]:
% - .[announcement]

В~мире живет много-много людей.
Люди рождаются, умирают, меняются.
А~еще в~мире живет много-много языков.
Языки тоже рождаются, умирают, меняются.
А~еще языки конфликтуют, влияют друг на~друга и~надстраиваются.

Бывают языки с~богатой родословной и~языки совсем без родственников.
Бывают языки, у~которых родственники были, но~умерли, и~языки с~родственниками
на~другом континенте.
А~еще иногда языки живут рядом, но~родственниками не~являются.
Да-да, живут, потому что языки живые.
Все языки, как люди, очень похожи, но~разные.

Изучать языки~--- это как решать математические задачи:
смотреть, что о~языке уже известно, думать, искать закономерности, порой что-то
считать и~находить ответы.
Если вы хотите узнать, почему языки одновременно такие похожие и~такие разные,
такие логичные и~нелогичные, порешать лингвистические задачи и~узнать, как
изучение языка помогает гуглу и~медицине, то~жду вас на~своем спецкурсе.

