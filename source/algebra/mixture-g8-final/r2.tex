% $date: 2016-06-12
% $timetable:
%   g8r2:
%     2016-06-12:
%       1:

\worksheet*{Финальный разнобой}

% $authors:
% - Владимир Алексеевич Брагин

\begin{problems}

\item
Можно~ли квадрат представить в~виде суммы 17 квадратов?

\item
Докажите, что для любого натурального $n$, взаимно простого с~$1001$,
выполнено сравнение $n^{60} \equiv 1 \pmod{1001}$.

\item
Докажите, что для любого натурального~$n$ выполнено
$n^{20} \equiv n^{4} \pmod{4080}$.

\item
Пусть $n$~--- нечетное число.
Докажите, что $(2^{n!} - 1)$ делится на~$n$.

\item
Дано простое число~$p$.
Для каких натуральных~$n$ число $n (n + p)$ является полным квадратом?

\item
\subproblem
Чему может быть равен $(x - 1, x^2 + x + 1)$?
\\
\subproblem
Число~$x$ таково, что $(x^3 - 1)$ является точным квадратом.
Докажите, что $(x - 1)$ делится на~$3$.

\item
Существует~ли такое 10000-значное число, что при замене любых трех его соседних
цифр, получится составное число?

\item
Пусть $p$~--- простое число.
\\
\subproblem
Докажите, что для любого~$a$ от~$1$ до~$(p - 1)$ существует ровно один
обратный остаток.
\\
\subproblem
Докажите, что если $x^2 \equiv 1 \pmod{p}$, то~$x \equiv \pm 1 \pmod{p}$.
\\
\subproblem\claim{Теорема Вильсона}
Докажите, что $(p - 1)! \equiv -1 \pmod{p}$.
\\
\subproblem
Докажите, что если для $m > 1$ имеет место сравнение
$(m - 1)! \equiv -1 \pmod{m}$, то~$m$~--- простое.

\item
Докажите, что любое простое число может быть делителем числа вида
$2^n + 3^n + 6^n - 1$.

\end{problems}

