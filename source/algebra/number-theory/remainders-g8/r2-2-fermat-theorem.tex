% $date: 2016-06-03
% $timetable:
%   g8r2:
%     2016-06-03:
%       2:

\worksheet*{Малая теорема Ферма}

% $authors:
% - Григорий Александрович Мерзон

\begin{problems}

\item
Пусть $a$ не~делится на~простое число~$p$.
\\
\subproblem
Докажите, что если $a x = a y \pmod{p}$, то~$x = y \pmod{p}$.
\\
\subproblem
Докажите, что числа $a$, $2 a$, $3 a$, \ldots, $(p - 1) a$ дают все ненулевые
остатки по~модулю~$p$.

\item\emph{(Малая теорема Ферма)}
Пусть $a$ не~делится на~простое число~$p$.
Докажите, что $a^{p-1} = 1 \pmod{p}$.
\emph{(Указание: докажите, что $(p - 1)! = a^{p-1} (p - 1)!$.)}

\item
Найдите остаток от~деления
\\
\subproblem $7^{100}$ на~$11$;
\qquad
\subproblem $7^{222}$ на~$23$;
\qquad
\subproblem $14^{14^{14}}$ на~$17$.

\item
\subproblem
Найдите остаток от~деления числа, состоящего из~$19$ девяток, на~$19$.
\\
\subproblem
Найдите число из~одних единиц, которое делилось~бы на~$2017$.

\item
\subproblem
Если $a$ не~делится на~$17$, то~либо $a^{8} - 1$, либо $a^{8} + 1$ делится
на~$17$.
\\
\subproblem
Пусть $x^2 = 1 \pmod{n}$.
Можно~ли утверждать, что $x = \pm 1 \pmod{n}$?

\item
Докажите, что $n^{561} = n \pmod{561}$ для любого $n$.

\end{problems}

