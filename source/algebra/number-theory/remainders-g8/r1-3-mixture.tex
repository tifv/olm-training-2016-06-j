% $date: 2016-06-05
% $timetable:
%   g8r1:
%     2016-06-05:
%       1:

\worksheet*{Вокруг МТФ и другие задачи}

% $authors:
% - Григорий Александрович Мерзон

\begingroup
    \ifdefined\mathup
        \def\eulerphi{\mathup{\phi}}%
    \else
        \def\eulerphi{\upphi}%
    \fi

Если $a$ взаимно просто с~$m$, то~$a^{\eulerphi(m)} = 1 \pmod{m}$, где
$\eulerphi(m)$~--- число остатков, взаимно простых с~$m$.

\begin{problems}

\itemy{0}
\subproblem
$\eulerphi(n m) = \eulerphi(n) \cdot \eulerphi(m)$, если $n$ и~$m$ взаимно
просты.
\\
\subproblem
Найдите $\eulerphi(p^{k})$, где $p$~--- простое.

\end{problems}

\hrulefill

\begin{problems}

\item
Докажите, что уравнение
\\
\subproblem $3 x^2 + 2 = y^2$
\qquad
\subproblem $8 x^3 - 13 y^3 = 17$
\\
не~имеет решений в~целых числах.
\emph{(Указание: рассмотрите остатки по~подходящему модулю.)}

\item
Докажите, что существует бесконечно много чисел, которые не~представимы в~виде
суммы
\quad
\subproblem двух
\quad
\subproblem трех
\quad
квадратов.

\item
\subproblem
Докажите, что если число $a^{n} - 1$ простое ($n > 1$, $a > 1$), то~$a = 2$,
а~$n$~--- простое число.
\\
\subproblem
Докажите, что если число $a^{n} + 1$ простое ($n > 1$, $a > 1$), то~$a$ четно,
а~$n$~--- степень двойки.

\item
Если число вида $2^{p} - 1$ составное, то~все его простые делители имеют вид
$2 k p + 1$.

\item
Докажите, что числа вида $2^{2^n} + 1$ попарно взаимно просты;
выведите отсюда, что простых чисел бесконечно много.

\item
Существует~ли степень тройки, оканчивающаяся на~$000\,001$?

\item
\subproblem
Существует~ли такое $n$, что $n^2 + 1$ делится на~$103$?
\\
\subproblem
Существует~ли такое $n$, что $n^2 + n + 1$ делится на~$101$?

\end{problems}

\endgroup % \def\eulerphi

