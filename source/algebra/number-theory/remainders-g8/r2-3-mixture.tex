% $date: 2016-06-04
% $timetable:
%   g8r2:
%     2016-06-04:
%       1:

\worksheet*{Задачи про делимость}

% $authors:
% - Григорий Александрович Мерзон

\begin{problems}

\item
\subproblem
Докажите, что если число $a^{n} - 1$ простое ($n > 1$, $a > 1$), то~$a = 2$,
а~$n$~--- простое число.
\\
\subproblem
Докажите, что если число $a^{n} + 1$ простое ($n > 1$, $a > 1$), то~$a$ четно,
а~$n$~--- степень двойки.

\item
\subproblem
Докажите, что существуют $100$ идущих подряд составных чисел.
\\
\subproblem
Докажите, что существуют $100$ идущих подряд чисел, среди которых ровно одно
простое.

\item
\subproblem
Докажите, что числа вида $2^{2^n} + 1$ попарно взаимно просты.
\\
\subproblem
Выведите из~предыдущего пункта, что простых чисел бесконечно много.

\item
Докажите, что уравнение
\\
\subproblem $3 x^2 + 2 = y^2$
\qquad
\subproblem $8 x^3 - 13 y^3 = 17$
\\
не~имеет решений в~целых числах.
\emph{(Указание: рассмотрите остатки по~подходящему модулю.)}

\item
Докажите, что существует бесконечно много чисел, которые не~представимы в~виде
суммы
\quad
\subproblem двух
\quad
\subproblem трех
\quad
квадратов.

\item
\subproblem
При каких $n$ число $n^2 + 1$ делится на~$65$?
\\
\subproblem
Существует~ли такое $n$, что $n^2 + 1$ делится на~$103$?

\end{problems}

