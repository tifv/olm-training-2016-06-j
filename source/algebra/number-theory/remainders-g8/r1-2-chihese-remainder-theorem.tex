% $date: 2016-06-03
% $timetable:
%   g8r1:
%     2016-06-03:
%       1:

\worksheet*{Китайская теорема об остатках}

% $authors:
% - Григорий Александрович Мерзон

\begin{problems}

\item
Генерал построил солдат в~колонну по~4, но~при этом рядовой Иванов остался
лишним.
Тогда генерал построил солдат в~колонну по~5.
И~снова Иванов остался лишним.
Когда~же и~в~колонне по~6 Иванов оказался лишним, генерал посулил ему наряд вне
очереди, после чего в~колонне по~7 Иванов нашел себе место и~никого лишнего
не~осталось.
Сколько солдат могло быть у~генерала?
% 301 + …

\item
Если $a$ взаимно просто с~$n$, то~сравнение $a x = b \pmod{m}$ имеет ровно одно
решения в~остатках по~модулю~$m$.

\item
\subproblem
Докажите, что для любого набора попарно взаимно простых чисел
$m_{1}$, $m_{2}$, \ldots, $m_{k}$ найдется такое $x$, что
\[
    \left\{ \begin{aligned}
        x &= 1 \pmod{m_1} \\
        x &= 0 \pmod{m_2} \\
        x &= 0 \pmod{m_3} \\
          & \cdots         \\
        x &= 0 \pmod{m_k}
    \end{aligned} \right.
\]
\par
\subproblem
Докажите, что для любого набора попарно взаимно простых чисел
$m_{1}$, $m_{2}$, \ldots, $m_{k}$ и~любого набора чисел
$a_{1}$, $a_{2}$, \ldots, $a_{k}$ система
\[
    \left\{ \begin{aligned}
        x &= a_{1} \pmod{m_{1}} \\
        x &= a_{2} \pmod{m_{2}} \\
          &  \cdots\\
        x &= a_{k} \pmod{m_{k}}
    \end{aligned} \right.
\]
имеет ровно одно решение по~модулю $m_{1} m_{2} \ldots m_{k}$.

\item
Докажите, что существует 100 идущих подряд натуральных чисел, каждое из~которых
делится на~какой-нибудь куб.

\item
При каких целых $n$ число $n^2 + 1$ делится на~$65$?

\item
Докажите, что $n^{561} = n \pmod{561}$ для любого $n$.

\item
Сколько существует 10-значных чисел $n$ таких, что последние цифры числа~$n^2$
образуют число~$n$?

\end{problems}

