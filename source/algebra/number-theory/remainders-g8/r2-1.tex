% $date: 2016-06-02
% $timetable:
%   g8r2:
%     2016-06-02:
%       1:

\worksheet*{Сравнения по модулю}

% $authors:
% - Григорий Александрович Мерзон

%Говорят, что числа $a$ и~$b$ \emph{сравнимы по~модулю~$m$}, если $a - b$
%делится на~$m$; обозначение: $a = b \pmod{m}$.
%Два числа сравнимы по~модулю~$m$ тогда и~только тогда, когда они дают один
%и~тот~же остаток при делении на~$m$.

\begin{problems}

\itemy{0}
Найдите остаток от~деления
\\
\subproblem $9^{100}$ на~$8$;
\qquad
\subproblem $12^{100}$ на~$13$;
\qquad
\subproblem $2^{1001}$ на~$11$.

\item
Найдите остаток от~деления $1000\cdot1001\cdot1002\cdot1003$
\\
\subproblem на~$999$;
\qquad
\subproblem на~$1004$.

\item
\subproblem
Любое число сравнимо с~суммой его цифр по~модулю~$9$.
\\
\subproblem
Любое число сравнимо со~знакопеременной суммой его цифр по~модулю~$11$.

\item
Пусть $n$~--- нечетное число.
Найдите остаток от~деления на~$n$ суммы $1^3 + 2^3 + \ldots + (n - 1)^3$.

\item
\subproblem
Пусть $a^{n} = 0 \pmod{a + b}$.
Докажите, что $b^{n} = 0 \pmod{a + b}$.
\\
\subproblem
$p$, $q$~--- простые числа, причем $q = p + 2$.
Докажите, что $p^{q} + q^{p}$ делится на~$p + q$.

\item
На~Луне имеют хождение монеты достоинством в~1, 15 и~50 фертингов.
Незнайка отдал за~покупку несколько монет и~получил сдачу~--- на~одну монету
больше.
Какова наименьшая возможная цена покупки?
% mod 7

\item
\subproblem
Докажите, что при $p > 2$ числитель дроби
\[
    1 + \frac{1}{2} + \frac{1}{3} + \ldots + \frac{1}{p - 1}
\]
делится на~$p$ ($p$~--- простое).
\\
\subproblemx{*}
Докажите, что при $p > 3$ он делится даже на~$p^2$.

\end{problems}

