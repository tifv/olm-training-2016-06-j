% $date: 2016-06-06
% $timetable:
%   g8r2:
%     2016-06-06:
%       1:

\section*{Китайская теорема об остатках}

% $authors:
% - Владимир Алексеевич Брагин

\begin{problems}

\item
Генерал построил солдат в~колонну по~4, но~при этом рядовой Иванов остался
лишним.
Тогда генерал построил солдат в~колонну по~5.
И~снова Иванов остался лишним.
Когда~же и~в~колонне по~6 Иванов оказался лишним, генерал посулил ему наряд вне
очереди, после чего в~колонне по~7 Иванов нашел себе место и~никого лишнего
не~осталось.
Сколько солдат могло быть у~генерала?

\item
Сколько существует наборов из~9 подряд идущих четырёхзначных чисел таких, что
первое из~них делится на~10, второе -- на~9, третье~--- на~8, ...
последнее -- на~2?

\item
Если $a$ взаимно просто с~$n$, то~сравнение $ax=b\pmod m$ имеет ровно одно
решения в~остатках по~модулю~$m$.

\item
Докажите, что для любого набора попарно взаимно простых чисел
$m_{1}$, $m_{2}$, \ldots, $m_{k}$ найдется такое $x$, что
\[
\left\{ \begin{aligned}
    x & \equiv 1 \pmod{m_{1}}
; \\
    x & \equiv 0 \pmod{m_{2}}
; \\
    x & \equiv 0 \pmod{m_{3}}
; \\ & \cdots \\
    x & \equiv 0 \pmod{m_{k}}
. \end{aligned} \right.
\]

\item
Докажите, что для любого набора попарно взаимно простых чисел
$m_{1}$, $m_{2}$, \ldots, $m_{k}$ и~любого набора чисел
$a_{1}$, $a_{2}$, \ldots, $a_{k}$ система
\[
\left\{ \begin{aligned}
    x & \equiv a_{1} \pmod{m_{1}}
; \\
    x & \equiv a_{2} \pmod{m_{2}}
; \\ & \cdots \\
    x & \equiv a_{k} \pmod{m_{k}}
; \end{aligned} \right.
\]
имеет ровно одно решение по~модулю $m_{1} m_{2} \ldots m_{k}$.

\item
Сколько остатков могут давать квадраты при делении на~20?

\item
Какие числа дают остаток 97 при делении на~101 и~остаток 15 при делении на~23?

\item
Докажите, что существует 100 идущих подряд натуральных чисел, каждое из~которых
делится на~какой-нибудь куб.

\item
Докажите, что существуют 18 последовательных натуральных чисел, среди которых
нет числа, взаимно простого с~остальными.

\end{problems}

