% $date: 2016-06-07
% $timetable:
%   g8r1:
%     2016-06-07:
%       1:

\worksheet*{Общие делители}

% $authors:
% - Владимир Алексеевич Брагин

\begin{problems}

\itemy{0}
Может~ли произведение двух последовательных натуральных чисел быть полным
квадратом?

\item
Чему может быть равен наибольший общий делитель чисел $5 n + 13$ и~$3 n + 8$?

\item
Каков наибольший возможный общий делитель чисел $9 m + 7 n$ и~$3 m + 2 n$,
если числа $m$ и~$n$ не~имеют общих делителей, кроме единицы?

\item
Дано простое число~$p$.
Для каких натуральных~$n$ число $n (n + p)$ является полным квадратом?

\item
Может~ли произведение трех последовательных натуральных чисел быть точной
седьмой степенью?

\item
Найдите все простые~$p$, для которых $p^3 + 2 p^2 + 1$~--- степень четверки.

\item
Число $(x^3 - 1)$ является полным квадратом.
Докажите, что число~$x$ дает остаток~$1$ при делении на~$3$.

\end{problems}

\emph{Считаем степень вхождения каждого простого по~отдельности.}

\begin{problems}

\item
Для натуральных $a$, $b$, $c$ докажите, что
\[
    [a, b, c]
=
    \frac{a b c \cdot (a, b, c)}{(a, b) \cdot (a, c) \cdot (b, c)}
\; . \]

\item
Натуральные числа $a$, $b$, $c$ взаимно просты в~совокупности и~удовлетворяют
равенству $(b + c) \cdot a = b c$.
Докажите, что $b + c$~--- точный квадрат.

\item
Натуральные числа $m$, $n$ таковы, что $m^2 + n^2 + m$ кратно $m n$.
Докажите, что $m$~--- квадрат натурального числа.

\end{problems}

