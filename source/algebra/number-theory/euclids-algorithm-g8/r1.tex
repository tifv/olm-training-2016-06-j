% $date: 2016-06-08
% $timetable:
%   g8r1:
%     2016-06-08:
%       1:

\worksheet*{Алгоритм Евклида}

% $authors:
% - Владимир Алексеевич Брагин

\claim{Алгоритм Евклида}
Пусть $a$ и~$b$~--- целые числа, не~равные одновременно нулю,
и~последовательность чисел
$a > b > r_{1} > r_{2} > r_{3} > r_{4} > \ldots > r_{n}$
определена тем, что каждое $r_k$~--- это остаток от~деления предпредыдущего
числа на~предыдущее, а~предпоследнее делится на~последнее нацело, то~есть
\begin{align*}
    a &= b q_0 + r_1
\, ; \\
    b &= r_1 q_1 + r_2
\, ; \\
    r_{1} &= r_2q_2 + r_3
\, ; \\ & \mathrel{\mathresizeto{\vdots}{=}} \\
    r_{k-2} &= r_{k-1} q_{k-1} + r_{k}
\, ; \\ & \mathrel{\mathresizeto{\vdots}{=}} \\
    r_{n-2} &= r_{n-1} q_{n-1} + r_n
\, ; \\
    r_{n-1} &= r_{n} q_{n}
\, . \end{align*}
Тогда $(a, b) = r_{n}$.

\claim{Линейное представление НОД}
Для любых натуральных чисел $a$ и~$b$ существуют целые числа $n$ и~$m$ такие,
что $(a, b) = a n + b m$.

\begin{problems}

\itemy{0}
\subproblem
Докажите, что алгоритм Евклида действительно выдает наибольший общий делитель
двух чисел.
\\
\subproblem
Докажите, что линейное представление НОД существует.

\item
Найдите все натуральные числа, дающие остаток~15 при делении на~87 и~остаток~2
при делении на~38.

\item
Докажите, что дробь $\frac{12 n + 1}{30 n + 2}$ несократима.

\item
Натуральные числа $a$ и~$b$ взаимно просты.
Докажите, что $(a + b, a^2 + b^2) < 3$.

\item
Известно, что $(m, n) = 1$.
Какое наибольшее возможное значение может принимать $(2016 n + m, 2016 m + n)$?

\item
Найдите $(\underbrace{11 \ldots 1}_{m}, \underbrace{11 \ldots 1}_{n})$.

\item
Найдите $(2^{n} - 1, 2^{m} - 1)$.

\item
Верно~ли, что для любых натуральных $a$ и~$b$ найдутся такие натуральные $p$
и~$q$, что при любом натуральном~$n$ числа $p + a n$ и~$q + b n$ взаимно
просты?

\end{problems}

