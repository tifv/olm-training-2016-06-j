% $date: 2016-06-08
% $timetable:
%   g9r1:
%     2016-06-08:
%       2:

\worksheet*{Классика ТЧ}

% $authors:
% - Андрей Борисович Меньщиков

\begingroup
    \ifdefined\mathup
        \def\eulerphi{\mathup{\phi}}%
    \else
        \def\eulerphi{\upphi}%
    \fi
    \def\divides{\mathrel{\vert}}%

\claim{Малая теорема Ферма}
$p$~--- простое, а~$a$ на~$p$ не~делится.
Тогда $a^{p-1} \equiv 1 \pmod{p}$.

\claim{Функция Эйлера}
$\eulerphi(n)$~--- это количество чисел, не~превосходящих натурального $n$,
взаимно простых с~$n$.

\claim{Формула для функции Эйлера}
\( \displaystyle
    \eulerphi(n)
=
    n \cdot
    \prod\limits_{\substack{
        p_i \divides n_{\phantom{i}} \\
        p_i \in \mathbb{P}
    }}
        \left( 1 - \frac{1}{p_i} \right)
\).

\claim{Теорема Эйлера}
Натуральные $a$ и~$n$ взаимно просты.
Тогда $a^{\eulerphi(n)} \equiv 1 \pmod{n}$.

\claim{Теорема Вильсона}
$p$~--- простое число.
Тогда $(p - 1)! \equiv -1 \pmod{p}$.

\begin{problems}

%\item
%Докажите, что для составного числа $561$ справедлив аналог малой теоремы Ферма:
%если $(a, 561) = 1$, то~$a^{560} \equiv 1 \pmod{561}$.

\item
Для натурального $k$ докажите, что $(2^{3^k} + 1) \kratno 3^{k+1}$.

\item
Для простого $p$ докажите, что $(2^{2^p} - 4) \kratno (2^p - 1)$.

\item
Для натурального $n$ докажите, что\enspace
\(\displaystyle
    \sum_{\substack{
        d \divides n \\
        d \in \mathbb{N}
    }}
        \eulerphi(d)
=
    n
\).

\item
Найдите сумму всех правильных несократимых дробей со~знаменателем~$n$.

\item
Докажите, что для любого натурального~$n$ существует число с~суммой цифр $n$,
делящееся на~$n$.

\item
Найдите все натуральные $n$ такие, что $(n! + 8) \kratno (2 n + 1)$.

\item
Докажите, что к~десятичной записи числа $2^{2016}$ слева можно приписать
несколько цифр так, чтобы получилась степень двойки.

\item
Найдите наименьшее простое число $p$ такое, что число $(2^{120!} - 1)$ делится
на~$p$, но~не~делится на~$p^2$.

\item
Натуральные $a$ и~$b$ больше $1$, но~меньше $100$.
Докажите, что существует натуральное~$n$ такое, что число
$a^{2^n} + b^{2^n}$~--- составное.

\item
Между двумя единицами пишется двойка, затем между любыми двумя соседними
числами пишется их сумма, и~т.~д.
Докажите, что каждое натуральное $n > 1$ в~итоге будет выписано ровно
$\eulerphi(n)$ раз.

\end{problems}

\endgroup % \def\eulerphi \def\divides

