% $date: 2016-06-01
% $timetable:
%   g8r2:
%     2016-06-01: {}
%   g8r1:
%     2016-06-01: {}

\worksheet*{Вступительная олимпиада}

\subsection*{Довывод}

\begin{problems}

\item
Сколько существует пятизначных чисел, кратных $101$ и~одинаково читающихся
слева направо и~справа налево?

\item
На~шахматной доске ($8 \times 8$) стоят $16$~королей, не~бьющих друг друга.
Какое наименьшее число королей при этом может стоять у~края доски?

\item
На~гипотенузе~$AB$ прямоугольного треугольника $ABC$ выбрана точка~$K$, для
которой $CK = BC$.
Отрезок~$CK$ пересекает биссектрису~$AL$ в~ее середине.
Найдите углы треугольника $ABC$.

\item
Решите уравнение в~натуральных числах $m (1 + m + m^2) = 4 n (n + 1)$.

\end{problems}

\subsection*{Вывод}

\begin{problems}

\item
В~треугольнике $ABC$ точка~$M$~--- середина $AC$.
На~стороне~$BC$ взяли точку~$K$ так, что угол $BMK$ прямой.
Оказалось, что $BK = AB$.
Найдите $\angle MBC$, если $\angle ABC = 110^{\circ}$.

\item
Дано $41$ различное натуральное число, меньшее $1000$.
Известно, что среди любых трех из~них есть два, дающих в~произведении точный
квадрат.
Докажите, что среди этих чисел есть точный квадрат.

\item
В~клетки таблицы $100 \times 100$ расставили числа от~$1$ до~$100$, при этом
каждое число оказалось написано ровно $100$ раз.
Докажите, что либо в~какой-то строчке, либо в~каком-то столбце не~менее
$10$ различных чисел.

\end{problems}

