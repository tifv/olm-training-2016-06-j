% $date: 2016-06-12
% $timetable:
%   g8r1:
%     2016-06-12:
%       1:

\worksheet*{Комбинаторика}

% $authors:
% - Ольга Дмитриевна Телешева
% - Андрей Юрьевич Кушнир

\begin{problems}

\item
В~несуществующем языке $8$ гласных и~$20$ согласных звуков.
Словом считается такая последовательность звуков, где гласные и~согласные
чередуются.
Сколько семибуквенных слов в~языке?

\item
Хромая ладья умеет ходить на~одну клетку вправо или на~одну клетку вверх
(по~клетчатой плоскости).
Ей очень хочется добраться из~клетки с~координатами $(0, 0)$ в~клетку
с~координатами $(m, n)$.
В~клетке $(p, q)$ стоит Гэндальф ($0 < p < m, 0 < q < n$).
Сколькими способами она может добраться до~цели, не~встречаясь с~Гэндальфом?

\item
Докажите тождества хотя бы двумя способами.
(В вашем арсенале:
комбинаторные конструкции, бином Ньютона, свойство треугольника Паскаля,
траектории хромой ладьи, явная формула для $C_{n}^{k}$\ldots)
\\[0.3ex]
\subproblem
$C_{n+1}^{k+1} = C_{n}^{k+1} + C_{n}^{k}$;
\\[0.5ex]
\subproblem
$k \cdot C_{n}^{k} = n \cdot C_{n-1}^{k-1}$;
\\[0.5ex]
\subproblem
$C_{n}^{0} + C_{n}^{1} + \ldots + C_{n}^{n-1} + C_{n}^{n} = 2^{n}$;
\\[0.5ex]
\subproblem
$C_{n}^{0} - C_{n}^{1} + C_{n}^{2} - \ldots + (-1)^n C_{n}^{n} = 0$;
\\[0.5ex]
\subproblem
$C_{r}^{m} \cdot C_{m}^{k} = C_{r}^{k} \cdot C_{r-k}^{m-k}$,
где $0 \leq k \leq m \leq r$;
%\\[0.5ex]
%\subproblem
%\(
%    C_{n}^{0} C_{n}^{n} + C_{n}^{1} C_{n}^{n-1}
%    + \ldots +
%    C_{n}^{n} C_{n}^{0}
%=
%    C_{2n}^{n}
%\);
\\[0.5ex]
\subproblem
$C_{n}^{n} + C_{n+1}^{n} + \ldots + C_{n+m}^{n} = C_{m+n+1}^{n+1}$;
\\[0.5ex]
\subproblem
\(
    0 \cdot C_{n}^{0} + 1 \cdot C_{n}^{1} + 2 \cdot C_{n}^{2} +
    3 \cdot C_{n}^{3} + \ldots + n \cdot C_{n}^{n}
=
    n \cdot 2^{n-1}
\);
\\[0.5ex]
\subproblem
$C_{n}^{0} + C_{n-1}^{1} + C_{n-2}^{2} + \ldots = F_{n+1}$,
где $F_{n}$~--- последовательность Фибоначчи ($F_{0} = 0$, $F_{1} = 1$);
\\[0.5ex]
\subproblem
$(C_{n}^{0})^2 + (C_{n}^{1})^2 + \ldots + (C_{n}^{n})^2 = C_{2n}^{n}$.

\item
У~Максима есть по~$5$ шоколадок пяти видов, он хочет их раздать по~одной
каждому из~$25$ детей, пришедших на~кружок.
Сколькими способами он может это сделать?

\item
Хромая ладья умеет ходить на~одну клетку вверх, вправо или вперед
по~трехмерному клетчатому пространству.
Сколькими способами она может попасть из~клетки $(0, 0, 0)$ в~клетку
$(m, n, k)$?

\item
Найдите число прямоугольников, составленных из~клеток доски с~$m$ горизонталями
и~$n$ вертикалями, которые содержат клетку с~координатами $(p, q)$.

\item
Сколькими способами можно построить замкнутую $n$-звенную ломаную, вершинами
которой являются вершины правильного $n$-угольника
(ломаная может быть самопересекающейся)?

\item
Сколькими способами можно выписать в~ряд числа от~$1$ до~$n$ в~некотором
порядке, чтобы для каждого числа $i$, стоящего не~в~самом начале, хотя~бы одно
из~чисел $i - 1$, $i + 1$ оказалось левее?

\end{problems}

