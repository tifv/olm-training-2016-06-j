% $date: 2016-06-11
% $timetable:
%   g8r2:
%     2016-06-11:
%       1:

\section*{Комбинаторика}

% $authors:
% - Ольга Дмитриевна Телешева

\begin{problems}

\item
Сколькими способами можно построить замкнутую ломаную, вершинами которой
являются вершины правильного шестиугольника
(ломаная может быть самопересекающейся)?

\item
На~плоскости дано $n$~прямых \emph{общего положения,} т.~е. таких, что никакие
две из~них не~параллельны и~никакие три не~пересекаются в~одной точке.
Чему равно число образованных ими треугольников?

\item
Найдите число прямоугольников, составленных из~клеток доски с~$m$ горизонталями
и~$n$ вертикалями, которые содержат клетку с~координатами $(p, q)$.

\item
\subproblem
У~Супермена есть 8 уникальных способностей, а~у~Человека-Паука только 7.
Каждый день они меняются одной способностью.
Сколькими способами могут произойти обмены в~течение трех дней?
\\
\subproblem
Бэтмен знает 8 новых анекдотов, а~Шапокляк~--- 7 старых.
Сколько различных способов обменять три анекдота одного на~три анекдота другой?

\item
План города представляет собой квадрат $n \times n$.
Улицы называются 0-я вертикальная, \ldots, $n$-ая вертикальная,
0-я горизонтальная, \ldots, $n$-ая горизонтальная,
на~пересечении $k$-ой горизонтальной улицы и~$m$-ой вертикальной улицы
находится автозаправка.
На~каждом перекрестке можно ехать влево или вверх.
Сколькими способами можно составить маршрут из~$X (0, 0)$ в~$Y (n, n)$, если
известно, что маршрут проходит через автозаправку?

\item
Имеется куб размером $10 \times 10 \times 10$, состоящий из~маленьких единичных
кубиков.
В~центре~$O$ одного из~угловых кубиков сидит кузнечик.
Он может прыгать в~центр кубика, имеющего общую грань с~тем, в~котором кузнечик
находится в~данный момент, причем так, чтобы расстояние до~точки
О~увеличивалось (то~есть, либо вверх, либо вправо, либо вперед).
Сколькими способами кузнечик может допрыгать до~кубика, противоположного
исходному?

\item
Рассмотрим множество~$A$ из~$n$ элементов.
Выделим в~нем элемент~$a$.
\\
\subproblem
Сколько существует подмножеств из~$k$ элементов, не~содержащих элемент~$a$?
\\
\subproblem
Сколько существует подмножеств из~$k$ элементов, содержащих элемент~$a$?
\\
\subproblem
Сколько существует подмножеств из~$k$ элементов, если нам не~важно, содержат
они или нет элемент~$a$?
\\
\subproblem
Докажите тождество $C_{n+1}^{k} = C_{n}^{k} + C_{n}^{k-1}$ двумя способами
(комбинаторно и~алгебраически).

\item
Докажите тождество $k \cdot C_{n}^{k} = n \cdot C_{n-1}^{k-1}$ двумя способами
(комбинаторно и~алгебраически).

\item
\subproblem
Сколькими способами можно выбрать произвольное непустое подмножество
из множества, содержащего $n$~элементов?
\par
Докажите формулы:
\\[0.3ex]
\subproblem
\(
    C_{n}^{0} + C_{n}^{1} + \ldots + C_{n}^{n-1} + C_{n}^{n}
=
    2^{n}
\);
\\[0.5ex]
\subproblem
\(
    C_{n}^{0} - C_{n}^{1} + C_{n}^{2} - \ldots + (-1)^{n} C_{n}^{n}
=
    0
\).

\item
Докажите формулы:
\\[0.3ex]
\subproblem
\(
    C_{r}^{m} \cdot C_{m}^{k}
=
    C_{r}^{k} \cdot C_{r-k}^{m-k}
\), где $0 \leq k \leq m \leq r$;
\\[0.5ex]
\subproblem
\(
    C_{n}^{k}
=
    C_{n-2}^{k} + 2 \cdot C_{n-2}^{k-1} + C_{n-2}^{k-2}
\);
\\[0.5ex]
\subproblem
\(
    C_{n}^{k}
=
    C_{n-3}^{k} + 3 \cdot C_{n-3}^{k-1} + 3 \cdot C_{n-3}^{k-2} + C_{n-3}^{k-3}
\);
\\[0.5ex]
\subproblem
\(
    C_{n}^{0} C_{n}^{n} + C_{n}^{1} C_{n}^{n-1} + \ldots + C_{n}^{n} C_{n}^{0}
=
    C_{2n}^{n}
\);
\\[0.5ex]
\subproblem
\(
    (C_{n}^{0})^{2} + (C_{n}^{1})^{2} + \ldots + (C_{n}^{n})^{2}
=
    C_{2n}^{n}
\);
\\[0.5ex]
\subproblem
\(
    C_{n}^{n} + C_{n+1}^{n} + \ldots + C_{n+m}^{n}
=
    C_{m+n+1}^{n+1}
\).

\end{problems}

