% $date: 2016-06-02
% $timetable:
%   g9r1:
%     2016-06-02:
%       2:

\worksheet*{Веса}

% $authors:
% - Глеб Александрович Погудин

\begin{problems}

\item
У~каждого депутата парламента Табулистана не~больше $20$~друзей среди
депутатов.
Их случайно делят на~Большую и~Малую палаты.
Хочется, чтобы у~каждого депутата Большой палаты было не~больше $15$~друзей
в~палате, а~у~каждого депутата Малой~--- не~больше пяти.
Каждого депутата, нарушающего это правило, перемещают в~другую палату.
Докажите, что этот процесс однажды завершится.

\item
Есть~$k$~тарелок, в~общей сложности в~них лежит $\frac{n (n + 1)}{2}$ орехов.
Каждым ходом мы достаем из~каждой тарелки по~ореху, заводим новую тарелку,
и~кладем орехи туда.
Если после какого-то хода одна из~тарелок опустела~--- выкидываем ее.
Докажите, что после некоторого количества ходов у~нас будет $n$~тарелок,
в~которых будет $1$, $2$, \ldots, $n$ орехов, соответственно.

\item
В~компании $n$~человек, у~некоторых есть по~нескольку монет.
Каждую минуту некоторый человек, имеющий не~менее $(n - 1)$ монет, может дать
каждому из~остальных по~одной монете.
Известно, что они могут действовать так, что эти жесты доброй воли будут
повторяться бесконечно долго.
Какое минимальное число монет могло быть в~сумме?

\item
В~точке с~координатами $(0, 0)$ лежит фишка.
За~один ход можно забрать фишку из~точки $(i, j)$ и~положить по~одной фишке
в~$(i + 1, j)$ и~в~$(i, j + 1)$.
При этом, ни в какой точке не должно быть более одной фишки.
Докажите, что в~любой момент времени хотя~бы в~одной клетке с~суммой координат
не~более трех будет фишка.

\item
В~ряд лежат $n$~спичек, все фосфором вверх.
За~ход можно взять спичку, лежащую фосфором вверх, не~лежащую с~краю, убрать
её, а~две соседние перевернуть.
Докажите, что оставить две спички можно тогда и~только тогда, когда $(n - 1)$
не~делится на~$3$.

\end{problems}

