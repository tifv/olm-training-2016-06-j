% $date: 2016-06-08
% $timetable:
%   g8r2:
%     2016-06-08:
%       1:

\worksheet*{Графы: пути, связность и остовные деревья}

% $authors:
% - Андрей Юрьевич Кушнир

\claim{Определения}
\emph{Маршрутом} длины~$k$ в~графе называется последовательность
$v_{0}$, $v_{1}$, $v_{2}$, \ldots, $v_{k}$ не~обязательно различных вершин
графа, такая что все пары вершин
$(v_{0}, v_{1})$, $(v_{1}, v_{2})$, $(v_{2}, v_{3})$, \ldots,
$(v_{k-1}, v_{k})$ соединены ребрами.
Маршрут называется \emph{путем} (или \emph{цепью}), если все его ребра
различны.
Маршрут называется \emph{простым путем} (или \emph{простой цепью}), если все
его вершины различны.
Цепь, у~которой первая и~последняя вершины совпадают, называется \emph{циклом}.
Цикл называется \emph{простым циклом}, если никакие две его вершины,
за~исключением первой и~последней, не~совпадают.
\emph{Расстоянием} между двумя вершинами графа называется минимальная длина
маршрута, соединяющего эти две вершины.

Графы предполагаются без петель и~кратных ребер (если не~оговорено иное).

\begin{problems}

\item
\emph{(Эта задача должна настроить вас на~необходимый уровень строгости.)}
В~графе две вершины можно соединить маршрутом.
Докажите, что их можно соединить простой цепью.

%\item
%Ребра полного графа раскрашены в~два цвета: красный и~синий.
%Докажите, что хотя~бы один из~двух графов (с~красными ребрами или
%с~синими ребрами) связен.

%\item
%Какое максимальное число ребер может быть в~несвязном графе на~$n$ вершинах?

\item
Напомним, что \emph{деревом} называется связный граф, не~содержащий простых
циклов.
\\
\subproblem
Докажите, что в~дереве любые две вершины соединяет единственная простая цепь.
\\
\subproblem
Докажите, что в~дереве всегда есть висячая вершина
(т.\,е. вершина степени~$1$).
\\
\subproblem
Докажите, что в~дереве на~$n$ вершинах ровно $n - 1$ ребер.

\item
\subproblem
Докажите, что в~связном графе можно выкинуть несколько ребер так, чтобы
осталось дерево (такое дерево называется \emph{остовным деревом} графа).
\\
\subproblem
Существует~ли граф, у~которого ровно два остовных дерева?

\item
Докажите, что у~каждого связного графа существует вершина, после удаления
которой он не~теряет связности.

\item
Докажите, что в~связном графе на~$n \geq 3$ вершинах существует маршрут длины
$2 n - 4$, проходящий через все вершины.

\item
Для каждой вершины графа напишем на~ней наибольшее из~расстояний до~всех
остальных вершин графа.
После этого выделим множество вершин, для которых выписанное число минимально.
Это множество называется \emph{центром} графа.
Докажите, что центр дерева состоит из~одной или двух вершин, соединенных
ребром.

\item
Есть несколько городов, одновременно существующих в~двух мирах: реальном
и~параллельном.
Два города соединены дорогой в~реальном мире тогда и~только тогда, когда они
они не~соединены в~параллельном.
Известно, что существуют два города, расстояние в~реальном мире между которыми
больше двух.
Докажите, что расстояние между любыми двумя городами в~параллельном мире
не~больше трех.

\item
В~связном графе $n \geq 5$ вершин и~$2 n - 1$ ребер.
Докажите, что в~нем можно найти простой цикл, после уничтожения всех ребер
которого граф не~потеряет связность.

%\item
%В~графе степень каждой вершины не~менее $10$.
%Докажите, что в~нем существует простая цепь длины~$10$
%(т.\,е. состоящая из~не~менее чем $11$~вершин).

\item
В~тридевятом государстве некоторые города соединены дорогами с~односторонним
движением.
В~стране транспортные проблемы: неверно, что из~любого города можно добраться
до~любого другого.
Царь Горох выбрал пару городов, еще не~соединенных дорогой, и~приказал
Ивану-дурачку соединить их.
Докажите, что Иван может задать направление движения так, что транспортные
проблемы в~стране не~решатся.

%\item
%\emph{Диаметром} графа называется максимальное расстояние между его вершинами.
%Какое максимальное число вершин может быть в~дереве с~$10$ висячими вершинами
%и~с~диаметром $10$?

%\item
%Несколько деревень соединены дорогами, причем длина каждой дороги меньше
%$10\,\text{км}$.
%Известно, из~любой деревни до~любой другой можно добраться, проехав меньше
%$10\,\text{км}$.
%Одну дорогу закрыли, но~все еще можно добраться из~любой деревни до~любой
%другой.
%Докажите, что это можно сделать, проехав меньше $30\,\text{км}$.
%(Дороги могут быть извилистыми, т.\,е. неравенство треугольника не~обязательно
%выполнено).

\end{problems}

