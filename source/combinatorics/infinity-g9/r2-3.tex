% $date: 2016-06-04
% $timetable:
%   g9r2:
%     2016-06-04:
%       2:

\section*{И вот ещё бесконечность}

% $authors:
% - Глеб Александрович Погудин

\begin{problems}

\item
Каждая точка плоскости, имеющая целочисленные координаты, раскрашена в~один
из~$n$ цветов.
Докажите, что найдется прямоугольник с~вершинами в~точках одного цвета.

\item
За~дядькой Черномором выстроилось чередой бесконечное число богатырей.
Докажите, что он может приказать части из~них выйти из~строя так, чтобы в~строю
осталось бесконечно много богатырей и~все они стояли по~росту (не~обязательно
в~порядке убывания роста).

\item
Имеется бесконечное количество карточек, на~каждой из~которых написано какое-то
натуральное число.
Известно, что для любого натурального числа~$n$ существуют ровно $n$~карточек,
на~которых написаны делители этого числа.
Докажите, что любое натуральное число встречается хотя~бы на~одной карточке.

\end{problems}

