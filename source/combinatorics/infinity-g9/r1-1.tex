% $date: 2016-06-06
% $timetable:
%   g9r1:
%     2016-06-06:
%       1:

\worksheet*{Бесконечность}

% $authors:
% - Глеб Александрович Погудин

\begin{problems}

\item
Натуральные числа покрасили в~$100$ цветов.
Докажите, что найдется цвет~$X$ такой, что есть бесконечно много пар чисел
цвета~$X$ на~расстоянии меньше $200$.

\item
Натуральные числа покрашены в~синий и~зеленый цвета, причем чисел каждого цвета
бесконечно много.
Докажите, что можно выбрать миллион синих и~миллион зеленых чисел так, что
сумма выбранных синих равна сумме выбранных зеленых.

\item
Имеется бесконечная шахматная доска.
Из~нее выкинули все клетки, у~которых обе координаты отрицательны или обе
координаты больше тысячи.
Можно~ли обойти такую фигуру (побывав в~каждой клетке ровно по~одному разу)
\\
\subproblem королем?
\qquad
\subproblem конем?

\item
Все натуральные числа покрасили в~несколько цветов.
Докажите, что найдется цвет такой, что для любого натурального~$n$ бесконечно
много чисел этого цвета делится на~$n$.

\item
В~некотором государстве какие-то слова объявлены \emph{плохими.}
Известно, что для любого $n$ найдется слово длины больше $n$, которое
не~содержит ни~одного плохого подслова.
Докажите, что найдется бесконечное слово, не~содержащее плохого подслова.

\item
Каждая точка плоскости, имеющая целочисленные координаты, раскрашена в~один
из~$n$ цветов.
Докажите, что найдется прямоугольник с~вершинами в~точках одного цвета.

\item
Натуральные числа покрашены в~несколько цветов.
Докажите, что найдутся цвет~$X$ и~число~$m$ такие, что для любого $k$ найдется
набор чисел $a_{1} < a_{2} < \ldots < a_{k}$ цвета~$X$, где
$a_{i+1} - a_{i} < m$.

\end{problems}

