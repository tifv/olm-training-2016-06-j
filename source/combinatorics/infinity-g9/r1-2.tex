% $date: 2016-06-07
% $timetable:
%   g9r1:
%     2016-06-07:
%       1:

\subsection*{Добавка}

% $caption: Добавка по бесконечности

% $authors:
% - Глеб Александрович Погудин

% $matter[-contained,no-header]:
% - verbatim: \setproblem{7}
% - verbatim: \worksheet*{Бесконечность}
% - .[contained]

\begin{problems}

\item
Каждая точка плоскости, имеющая целочисленные координаты, раскрашена в один
из $n$~цветов.
Докажите, что для любого $N$ найдется цвет~$X$ такой, что найдется точка
цвета~$X$, с которой на одной горизонтали бесконечно много точек цвета~$X$,
а на одной вертикали не меньше $N$ точек цвета~$X$.

\item
За~дядькой Черномором выстроилось чередой бесконечное число богатырей.
Докажите, что он может приказать части из~них выйти из~строя так, чтобы в~строю
осталось бесконечно много богатырей и~все они стояли по~росту (не~обязательно
в~порядке убывания роста).

\item
Имеется возрастающая последовательность натуральных чисел
$a_{1} < a_{2} < \ldots < a_{n} < \ldots$, где для всякого $k$ выполнено
$a_{k+1} - a_{k} < 10^6$.
Докажите, что найдутся различные $i$ и~$j$ такие, что $a_i$ делится на~$a_j$.

\end{problems}

