% $date: 2016-06-03
% $timetable:
%   g9r2:
%     2016-06-03:
%       1:

\worksheet*{Ещё бесконечность}
% $caption: Бесконечность. Добавка

% $authors:
% - Глеб Александрович Погудин

% $matter[contained,no-header,-fixproblem]:
% - verbatim: \resetproblem
% - .[fixproblem]

\begin{problems}

\item
Натуральные числа покрашены в~несколько цветов.
Докажите, что найдется цвет такой, что для любого $n$ среди чисел этого цвета
бесконечно много точных $n$-ых степеней.

\item
Имеется бесконечная шахматная доска.
Из~нее выкинули все клетки, у~которых обе координаты отрицательны или обе
координаты больше тысячи.
Можно~ли обойти такую фигуру (побывав в~каждой клетке ровно по~одному разу)
\\
\subproblem королем?
\qquad
\subproblem конем?
\\
\subproblem ладьей?\enspace
(Считается, что ладья <<прыгает>>, то~есть посещает только начальную и~конечную
клетки своего хода.)

\item
В~связном бесконечном графе все вершины имеют конечную степень.
Докажите, что есть бесконечный несамопересекающийся путь.

\item
В~связном бесконечном графе все вершины имеют конечную степень.
Докажите, что можно выкинуть часть ребер (возможно, бесконечно много) так,
чтобы между любыми двумя вершинами шел ровно один путь по~оставшимся ребрам.

\item
В~некотором государстве какие-то слова объявлены \emph{плохими.}
Известно, что для любого $n$ найдется слово длины больше $n$, которое
не~содержит ни~одного плохого подслова.
Докажите, что найдется бесконечное слово, не~содержащее плохого подслова.

\end{problems}

