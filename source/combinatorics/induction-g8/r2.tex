% $date: 2016-06-05
% $timetable:
%   g8r2:
%     2016-06-05:
%       1:

\worksheet*{Индукция и минимальный контрпример}

% $authors:
% - Андрей Юрьевич Кушнир

\begin{problems}

\item
Есть колода из~$n$ карточек, пронумерованных числами от~$1$ до~$n$.
За~раз разрешается выбрать стопку из~несколько подряд идущих карточек в~любом
месте колоды, перевернуть эту стопку и~положить обратно.
Докажите, что за~несколько раз можно упорядочить все карточки в~колоде
по~возрастанию.

\item
В~каждой клетке доски $3 \times n$ стоит фишка одного из~трех цветов, причем
всего фишек каждого цвета на~доске поровну.
Внутри каждой из~трех строк разрешается переставлять фишки в~любом порядке.
Докажите, что их можно расставить так, что в~каждом столбце будут фишки всех
трех цветов.

\item
У~выпуклого многоугольника наружу растут волосы.
Проведены несколько его диагоналей, причем у~каждой диагонали с~одной стороны
растут волосы.
Докажите, что среди частей, на~которые разрезан многоугольник этими
диагоналями, найдется такая, волосы которой растут наружу.

\item
Автомобиль с~заполненным баком может проехать все кольцевое шоссе целиком.
Вдоль шоссе расположены несколько бензоколонок, суммарного количества бензина
в~которых хватит, чтобы заполнить бак.
Докажите, что существует бензоколонка, стартовав с~которой с~пустым баком,
можно проехать все шоссе (дозаправляясь по~пути).

\item
В~выпуклом многоугольнике проведено несколько попарно непересекающихся
по~внутренним точкам диагоналей
(из~одной вершины могут выходить несколько диагоналей).
Докажите, что найдутся хотя~бы две вершины многоугольника, из~которых
не~выходит ни~одной диагонали.

\item
В~группе людей каждый хоть с~кем-то из~этой группы знаком.
Докажите, что группу можно разбить на~две подгруппы с~условием, чтобы у~любого
человека был хотя~бы один знакомый из~противоположной подгруппы.

\item
Клетки шахматной доски $100 \times 100$ раскрашены в~$4$ цвета таким образом,
что в~любом квадрате $2 \times 2$ все клетки разного цвета.
Докажите, что угловые клетки раскрашены в~разные цвета.

\item
Вершины выпуклого многоугольника раскрашены в~три цвета так, что каждый цвет
присутствует и~никакие две соседние вершины не~окрашены в~один цвет.
Докажите, что многоугольник можно разбить диагоналями на~треугольники
с~условием, чтобы у~каждого треугольника вершины были трех разных цветов.

\item
В~чемпионате по~футболу участвует $n$~команд.
В~какое минимальное число дней его можно организовать так, чтобы каждая команда
сыграла с~каждой и~чтобы никакой команде не~пришлось сыграть две игры в~один
день?

\end{problems}

