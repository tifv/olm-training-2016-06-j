% $date: 2016-06-04
% $timetable:
%   g9r1:
%     2016-06-04:
%       1:

\worksheet*{Ещё аддитивные задачи}
% $caption: Аддитивная комбинаторика. Добавка

% $authors:
% - Глеб Александрович Погудин

% $matter[contained,no-header,-fixproblem]:
% - verbatim: \resetproblem
% - .[fixproblem]

\begin{problems}

\item
Среди чисел от~$1$ до~$1000$ выбрали $455$.
Докажите, что среди выбранных найдутся два с~суммой кратной $20$.

\item
Даны различные натуральные числа $a_{1}$, \ldots, $a_{14}$.
На~доску выписаны все $196$ чисел вида $a_{k} + a_{l}$, где
$1 \leq k, l \leq 14$.
Может~ли оказаться, что для любой комбинации из~двух цифр среди написанных
на~доске чисел найдется хотя~бы одно число, оканчивающееся на~эту комбинацию
(то~есть, найдутся числа, оканчивающиеся на~$00$, $01$, $02$, \ldots, $99$)?

\item
Даны $n$ натуральных чисел с~суммой $s < 2 n$.
Докажите, что для любого натурального $m \leq s$, можно выбрать несколько
из~этих чисел так, чтобы их сумма была равна $m$.

\item
Даны $2000$ целых чисел с~суммой $1$, причем каждое число по~модулю
не~превосходит $1000$.
Докажите, что можно выбрать несколько из~этих чисел так, чтобы сумма выбранных
была равна нулю.

\item
На~доску выписано красным маркером $101$ натуральное число от~$1$ до~$10^{6}$.
Докажите, что можно выписать $100$ синих чисел из~того~же промежутка таких, что
все суммы $(\text{синее} + \text{красное})$ будут различны.

\end{problems}

