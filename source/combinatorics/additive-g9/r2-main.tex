% $date: 2016-06-06
% $timetable:
%   g9r2:
%     2016-06-06:
%       2:

\worksheet*{Сумма да тюрьмма (аддитивная комбинаторика)}
% $caption: Аддитивная комбинаторика

% $authors:
% - Глеб Александрович Погудин

\begin{problems}

\item
Выбрали $n$ чисел от~$1$ до~$1000$.
Докажите, что не~все попарные суммы различные (число можно складывать с~самим
собой), если
\\
\subproblem $n = 64$;
\qquad
\subproblem $n = 46$.

\item
Какое наибольшее количество различных чисел можно выбрать из~множества
$\{ 1, 2, \ldots, 9 \}$, чтобы все попарные суммы были различны?

\item
Среди чисел от~$1$ до~$1000$ выбрали $455$.
Докажите, что среди выбранных найдутся два с~суммой кратной $20$.

\item
Выбраны два множества натуральных чисел $A$ и~$B$, в~множестве~$A$ содержится
$n$~элементов, в~множестве~$B$~--- $m$.
Какое наименьшее количество элементов может содержать множество сумм вида
$a_{i} + b_{j}$, где $a_{i} \in A$ и~$b_{j} \in B$?

\item
На~доску выписано красным маркером $101$ натуральное число от~$1$ до~$10^6$.
Докажите, что можно выписать $100$ синих чисел из~того~же промежутка таких, что
все суммы $(\text{синее} + \text{красное})$ будут различны.

\item
Среди чисел от~$1$ до~$100$ выбрали некоторые $16$.
Докажите, что среди этих шестнадцати найдутся различные $a$, $b$, $c$, $d$
такие, что $a + b = c + d$.

\item
Среди чисел от~$1$ до~$501$ выбрали $250$.
Докажите, что
\\
\subproblem
среди них найдутся различные $a$, $b$, $c$, $d$ такие, что $a + b + c + d$
кратно $23$;
\\
\subproblem
для любого $x$ среди них найдутся различные $a$, $b$, $c$, $d$ такие, что
$a + b + c + d \equiv x \pmod{23}$.

\end{problems}

