% $date: 2016-06-03
% $timetable:
%   g9r1:
%     2016-06-03:
%       2:

\worksheet*{Фома Аквинский и немного аддитивной комбинаторики}

% $authors:
% - Глеб Александрович Погудин

\begingroup
    \def\abs#1{\lvert #1 \rvert}%

\begin{problems}

\item
Какое наибольшее количество различных чисел можно выбрать из~множества
$\{ 1, 2, \ldots, 9 \}$, чтобы все попарные суммы были различны?

\item
Выбраны два множества натуральных чисел $A$ и~$B$, причем $\abs{A} = n$,
$\abs{B} = m$.
Какое наименьшее количество элементов может содержать множество сумм вида
$a_{i} + b_{j}$, где $a_{i} \in A$ и~$b_{j} \in B$?

\item
Натуральные числа от~$1$ до~$100$ покрашены в~семь цветов.
Докажите, что найдется цвет, для которого среди попарных сумм
(число с~самим собой можно складывать) найдутся две равных.

\item
Среди чисел от~$1$ до~$100$ выбрали некоторые $16$.
Докажите, что среди этих шестнадцати найдутся различные $a$, $b$, $c$, $d$
такие, что $a + b = c + d$.

\item
Докажите, что для любого~$n$ есть такое множество натуральных чисел, что все
попарные суммы (можно складывать число с~самим собой) различные, и~среди этих
попарных сумм есть $n$ последовательных натуральных чисел.

\item
В~таблице $n \times n$ расставлены числа $0$, $1$ и~$-1$.
Оказалось, что все суммы по~строкам и~по~столбцам~--- это $2 n$ различных
чисел.
\\
\subproblem
Приведите пример такой таблицы для любого четного $n$.
\\
\subproblem
Докажите, что $n$ четно.

\item
Среди чисел от~$1$ до~$501$ выбрали $250$.
Докажите, что
\\
\subproblem
среди них найдутся различные $a$, $b$, $c$, $d$ такие, что $a + b + c + d$
кратно $23$;
\\
\subproblem
для любого $x$ среди них найдутся различные $a$, $b$, $c$, $d$ такие, что
$a + b + c + d \equiv x \pmod{23}$.

\item
Пусть есть набор из~$n$ натуральных чисел.
Докажите, что множество попарных сумм (можно складывать число с~самим собой)
содержит от~$(2 n - 1)$ до~$C_{n+1}^{2}$ элементов, причем все эти возможности
реализуются.

\end{problems} % \def\abs

\endgroup

