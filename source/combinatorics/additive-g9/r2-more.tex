% $date: 2016-06-07
% $timetable:
%   g9r2:
%     2016-06-07:
%       2:

\worksheet*{Добавка про суммы}
% $caption: Аддитивная комбинаторика. Добавка

% $authors:
% - Глеб Александрович Погудин

% $matter[-contained,no-header]:
% - verbatim: \setproblem{7}
% - .[contained]

\begin{problems}

\item
Даны $n + 1$ попарно различных натуральных числа, меньших $2 n$ ($n > 1$).
Докажите, что среди них найдутся три таких числа, что сумма двух из них равна
третьему.

\item
Даны $n$ натуральных чисел с~суммой $s < 2 n$.
Докажите, что для любого натурального $m \leq s$, можно выбрать несколько
из~этих чисел так, чтобы их сумма была равна $m$.

\item
Даны $2000$ целых чисел с~суммой $1$, причем каждое число по~модулю
не~превосходит $1000$.
Докажите, что можно выбрать несколько из~этих чисел так, чтобы сумма выбранных
была равна нулю.

\end{problems}

