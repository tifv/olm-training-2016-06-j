% $date: 2016-06-02
% $timetable:
%   g9r2:
%     2016-06-02:
%       1:

\section*{Том Сойер и натуральный ряд}

% $authors:
% - Глеб Александрович Погудин

\begin{problems}

\item
Некоторые натуральные числа покрасили в~зеленый цвет.
Любой конечный набор зеленых чисел имеет общий делитель больший единицы.
Докажите, что все зеленые числа делятся на~некоторое число большее единицы.

\item
Докажите, что в~бесконечной последовательности попарно различных натуральных
чисел, больших единицы, найдется бесконечное количество чисел, которые больше
своего номера в~этой последовательности.

\item
Натуральные числа покрасили в~$100$ цветов.
Докажите, что найдется цвет~$X$ такой, что есть бесконечно много пар чисел
цвета~$X$ на~расстоянии меньше $200$.

\item
Натуральные числа покрасили в~$100$ цветов.
Докажите, что найдется цвет~$X$ и~число $m < 200$ такие, что есть бесконечно
много пар чисел цвета~$X$ на~расстоянии ровно $m$.

\item
Натуральные числа покрашены в~синий и~зеленый цвета, причем чисел каждого цвета
бесконечно много.
Докажите, что можно выбрать миллион синих и~миллион зеленых чисел так, что
сумма выбранных синих равна сумме выбранных зеленых.

\item
Все натуральные числа покрасили в~несколько цветов.
Докажите, что найдется цвет такой, что для любого натурального~$n$ бесконечно
много чисел этого цвета делится на~$n$.

\item
Натуральные числа покрашены в~два цвета.
Докажите, что найдется одноцветная арифметическая прогрессия длины три.

\item
Натуральные числа покрашены в~несколько цветов.
Докажите, что найдутся цвет~$X$ и~число~$m$ такие, что для любого $k$ найдется
набор чисел $a_1 < a_2 < \ldots < a_k$ цвета~$X$, где $a_{i + 1} - a_i < m$.

\end{problems}

